\documentclass[a4paper, 11pt]{article}                  % paper and font size
\usepackage[top=1.1in, bottom=1.1in]{geometry}          % margins
\usepackage[protrusion=true,expansion=true]{microtype}  % better typography
\usepackage{../assets/problem_set}                   % imports style file
\usepackage{../assets/notation}                   % imports style file

% Set the colors for answer boxes in this document (default is olive green)
% \setanswerboxcolors{Gray!10}{Gray!80}             % For gray theme
% \setanswerboxcolors{blue!5}{blue!40}            % Blue theme
% \setanswerboxcolors{RawSienna!15}{RawSienna!50} % Orange theme  
% \setanswerboxcolors{Plum!10}{Purple!50}         % Purple theme

%----------------------------------------------------------------------------------------
%	TITLE
%----------------------------------------------------------------------------------------

\title{The Modern History of Modal Logic}  % Title
\pset{Problem Set 05: Due March 31st}              % Problem Set
\date{\today}                              % Date

%----------------------------------------------------------------------------------------

\begin{document}
\maketitle              % Print the title section
\thispagestyle{empty}   % Drop header and page number from the first page

%----------------------------------------------------------------------------------------

%%%%%%%%%%%%%%%%%%%%
%%% INSTRUCTIONS %%%
%%%%%%%%%%%%%%%%%%%%

% 1. Make sure to pull changes (sync) before starting to work on any problems
% 2. Add your name to the problems you want to work on, pushing changes immediately to "check out" the problems
% 3. If a problem has been checked out once, make a copy of the solution block below to contribute your solution below
% 4. If there are already two people working on a problem, try to find another problem to work on
% 5. If you make some progress (even partial) and get stuck, push changes for others to see/help/etc.
% 6. If you have questions or comments, please open a new issue (it's possible to link line numbers) 
% 7. It is best practice to have one sentence per numbered line, making it easier to add comments
% 8. If possibly, try to avoid breaking formal expressions across lines to improve readability
% 9. I will comment on your solutions, adding my name to the reviewers (feel free to add comments/questions as well)
% 10. If the comments have been addressed or no longer relevant, feel free to remove comments
% 11. If you want to add comments, make sure to create a new line below to do so (this will help avoid conflicts)
% 12. All the defined commands can be found in assets/notation.sty but feel free to use what you like
% 13. I will try to include comments alongside the definition of environments in notation.sty to explain their use
% 14. Make sure that the document builds without errors before pushing changes
% 15. Please try to choose a diversity of problem types to work on
% 16. To streamline your proofs, try to unpack claims with an existential flavor before those with a universal flavor
% 17. In using reductio proofs, think about what the easiest thing to negate would be given the intended conclusion
% 18. Sometimes a contraposition style proof is just as easy as a reductio, and sometimes direct is the best of all
% 19. Try to neither skip substantial steps nor to make any unforced moves
% 20. Cutting and pasting from your or other's proofs is perfectly fine as long as the end result is good

%%%%%%%%%%%%%%%%%%%%

\section{Cartesian Semantics}

\begin{enumerate}

	\item[\bf Countermodels:] Evaluate the following, providing a proof or $\BLC$ countermodel.

    \item \textit{If} $\Gamma \MLmodels[] \metaA$, \textit{then} $\Box\Gamma \MLmodels[] \Box\metaA$.
	      % \answer{ NAME % collaborator names
	      % }{% body of the argument
	      %   Your answer goes here
	      % }

    \item $\MLmodels[] \Box\metaA \rightarrow \metaA$.
	      % \answer{ NAME % collaborator names
	      % }{% body of the argument
	      %   Your answer goes here
	      % }

    \item $\MLmodels[] \Box\metaA \rightarrow \Box\Box\metaA$.
	      % \answer{ NAME % collaborator names
	      % }{% body of the argument
	      %   Your answer goes here
	      % }

    \item $\MLmodels[] \metaA \rightarrow \Box\Diamond\metaA$.
	      % \answer{ NAME % collaborator names
	      % }{% body of the argument
	      %   Your answer goes here
	      % }

    \item $\MLmodels[] \Box\metaA \rightarrow \sometimes\metaA$.
	      % \answer{ NAME % collaborator names
	      % }{% body of the argument
	      %   Your answer goes here
	      % }

    \item $\MLmodels[] \boxtimes\metaA \leftrightarrow \Box\always\metaA$.
	      % \answer{ NAME % collaborator names
	      % }{% body of the argument
	      %   Your answer goes here
	      % }

    \item $\MLmodels[] \boxtimes\metaA \rightarrow \metaA$.
	      % \answer{ NAME % collaborator names
	      % }{% body of the argument
	      %   Your answer goes here
	      % }

    \item $\MLmodels[] \boxtimes\metaA \rightarrow \boxtimes\boxtimes\metaA$.
	      % \answer{ NAME % collaborator names
	      % }{% body of the argument
	      %   Your answer goes here
	      % }

    \item $\MLmodels[] \metaA \rightarrow \boxtimes\diamondtimes\metaA$.
	      % \answer{ NAME % collaborator names
	      % }{% body of the argument
	      %   Your answer goes here
	      % }

    \item $\MLmodels[] \boxtimes\metaA \rightarrow \sometimes\metaA$.
	      % \answer{ NAME % collaborator names
	      % }{% body of the argument
	      %   Your answer goes here
	      % }

\end{enumerate}





\section{Task Semantics}

\begin{enumerate}

	\item[\bf Validity:] Evaluate the following, providing a proof or $\BL$ countermodel.

    \item $\MLmodels[] \Box\metaA \rightarrow \Box\Future\metaA$.
	      % \answer{ NAME % collaborator names
	      % }{% body of the argument
	      %   Your answer goes here
	      % }

    \item $\MLmodels[] \Box\metaA \rightarrow \always\metaA$.
	      % \answer{ NAME % collaborator names
	      % }{% body of the argument
	      %   Your answer goes here
	      % }

    \item $\MLmodels[] \Box\metaA \rightarrow \Future\Box\metaA$.
	      % \answer{ NAME % collaborator names
	      % }{% body of the argument
	      %   Your answer goes here
	      % }


    \item $\MLmodels[] \sometimes\metaA \rightarrow \Diamond\metaA$.
	      % \answer{ NAME % collaborator names
	      % }{% body of the argument
	      %   Your answer goes here
	      % }


\end{enumerate}





\section{Extensions}

\begin{enumerate}

	\item[\bf Regimentation:] Regiment the following in $\BL$ and $\BLP$ disambiguating as needed.

    \item If it could rain, it will eventually stop.
	      % \answer{ NAME % collaborator names
	      % }{% body of the argument
	      %   Your answer goes here
	      % }

    \item Either it will always eventually rain or it could eventually always rain.
	      % \answer{ NAME % collaborator names
	      % }{% body of the argument
	      %   Your answer goes here
	      % }

    \item If it has always been raining, then it will eventually stop.
	      % \answer{ NAME % collaborator names
	      % }{% body of the argument
	      %   Your answer goes here
	      % }

    \item Either it could always rain, or it has been raining and will rain again.
	      % \answer{ NAME % collaborator names
	      % }{% body of the argument
	      %   Your answer goes here
	      % }

    \item If it could eventually always rain, then it eventually it could always rain.
	      % \answer{ NAME % collaborator names
	      % }{% body of the argument
	      %   Your answer goes here
	      % }

    \item Either it has been raining or it will eventually rain.
	      % \answer{ NAME % collaborator names
	      % }{% body of the argument
	      %   Your answer goes here
	      % }

    \item If it could eventually stop raining, then it has been raining.
	      % \answer{ NAME % collaborator names
	      % }{% body of the argument
	      %   Your answer goes here
	      % }

    \item If it will be raining, then it could have always been raining.
	      % \answer{ NAME % collaborator names
	      % }{% body of the argument
	      %   Your answer goes here
	      % }

    \item If it has always been raining, then it could eventually stop.
	      % \answer{ NAME % collaborator names
	      % }{% body of the argument
	      %   Your answer goes here
	      % }

    \item If it could eventually always rain, then it will always rain.
	      % \answer{ NAME % collaborator names
	      % }{% body of the argument
	      %   Your answer goes here
	      % }

	\item[\bf Mixed Modals:] Evaluate the following, providing a proof or $\BLP$ countermodel.

    \item $\Inevitably\metaA \MLmodels[] \Openpast\metaA$.
	      % \answer{ NAME % collaborator names
	      % }{% body of the argument
	      %   Your answer goes here
	      % }

    \item $\Openpast\metaA \wedge \Openfuture\metaA \MLmodels[] \metaA$.
	      % \answer{ NAME % collaborator names
	      % }{% body of the argument
	      %   Your answer goes here
	      % }

    \item $\Will\metaA \MLmodels[] \Past\Will\metaA$
	      % \answer{ NAME % collaborator names
	      % }{% body of the argument
	      %   Your answer goes here
	      % }

    \item $\Inevitably\metaA \MLmodels[] \Openfuture\metaA$.
	      % \answer{ NAME % collaborator names
	      % }{% body of the argument
	      %   Your answer goes here
	      % }

    \item $\metaA \MLmodels[] \Openpast\metaA \wedge \Openfuture\metaA$.
	      % \answer{ NAME % collaborator names
	      % }{% body of the argument
	      %   Your answer goes here
	      % }

    \item $\could\metaA \MLmodels[] \Past\could\metaA$
	      % \answer{ NAME % collaborator names
	      % }{% body of the argument
	      %   Your answer goes here
	      % }


\end{enumerate}



%%% Bibliography %%%

% \vfill
% \begin{small} %%Makes bib small text size
%   \singlespacing %%Makes single spaced
%   \bibliographystyle{../../assets/bib_style} %%bib style found locally or in textmf/bibtex/bst
%   \setlength{\bibsep}{0.5pt} %%Changes spacing between bib entries
%   \bibliography{../../assets/modal_history} %%bib database found locally or in textmf/bibtex/bib
%   \thispagestyle{empty} %%Removes page numbers
% \end{small} %%End makes bib small text size

\end{document}
