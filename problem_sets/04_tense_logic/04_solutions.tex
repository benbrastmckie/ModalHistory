\documentclass[a4paper, 11pt]{article}                  % paper and font size
\usepackage[top=1.1in, bottom=1.1in]{geometry}          % margins
\usepackage[protrusion=true,expansion=true]{microtype}  % better typography
\usepackage{../assets/problem_set}                   % imports style file
\usepackage{../assets/notation}                   % imports style file

% Set the colors for answer boxes in this document (default is olive green)
% \setanswerboxcolors{Gray!10}{Gray!80}             % For gray theme
% \setanswerboxcolors{blue!5}{blue!40}            % Blue theme
% \setanswerboxcolors{RawSienna!15}{RawSienna!50} % Orange theme  
% \setanswerboxcolors{Plum!10}{Purple!50}         % Purple theme

%----------------------------------------------------------------------------------------
%	TITLE
%----------------------------------------------------------------------------------------

\title{The Modern History of Modal Logic}  % Title
\pset{Problem Set 04: Due March 17rd}              % Problem Set
\date{\today}                              % Date

%----------------------------------------------------------------------------------------

\begin{document}
\maketitle              % Print the title section
\thispagestyle{empty}   % Drop header and page number from the first page

%----------------------------------------------------------------------------------------

%%%%%%%%%%%%%%%%%%%%
%%% INSTRUCTIONS %%%
%%%%%%%%%%%%%%%%%%%%

% 1. Make sure to pull changes (sync) before starting to work on any problems
% 2. Add your name to the problems you want to work on, pushing changes immediately to "check out" the problems
% 3. If a problem has been checked out once, make a copy of the solution block below to contribute your solution below
% 4. If there are already two people working on a problem, try to find another problem to work on
% 5. If you make some progress (even partial) and get stuck, push changes for others to see/help/etc.
% 6. If you have questions or comments, please open a new issue (it's possible to link line numbers) 
% 7. It is best practice to have one sentence per numbered line, making it easier to add comments
% 8. If possibly, try to avoid breaking formal expressions across lines to improve readability
% 9. I will comment on your solutions, adding my name to the reviewers (feel free to add comments/questions as well)
% 10. If the comments have been addressed or no longer relevant, feel free to remove comments
% 11. If you want to add comments, make sure to create a new line below to do so (this will help avoid conflicts)
% 12. All the defined commands can be found in assets/notation.sty but feel free to use what you like
% 13. I will try to include comments alongside the definition of environments in notation.sty to explain their use
% 14. Make sure that the document builds without errors before pushing changes
% 15. Please try to choose a diversity of problem types to work on
% 16. To streamline your proofs, try to unpack claims with an existential flavor before those with a universal flavor
% 17. In using reductio proofs, think about what the easiest thing to negate would be given the intended conclusion
% 18. Sometimes a contraposition style proof is just as easy as a reductio, and sometimes direct is the best of all
% 19. Try to neither skip substantial steps nor to make any unforced moves
% 20. Cutting and pasting from your or other's proofs is perfectly fine as long as the end result is good




%%%%%%%%%%%%%%%%%%%%

\section{Regimentation}

\begin{enumerate}

  \item[\bf Negation:] Provide a definition of negation using the other operators in $\TL$.
	      % \answer{ NAME % collaborator names
	      % }{% body of the argument
	      %   Your answer goes here
	      % }

	\item[\bf Regimentation:] Regiment the following in $\TL$ and $\TLI$ disambiguating as needed.

    \item If it is raining, it will stop.
      \answer{ Ben % Collaborators
      }{% body of the argument
        Letting $R$ symbolize `It is raining', we may regiment the claim by $R \rightarrow \future\neg R$ in $\TL$.
        This asserts that if it is raining, there is a time in the actual future in which it stops raining.

        Alternatively, we may take `will' to convey that if it is raining, then for every future there is a time in which the rain stops.
        We may capture this reading in $\TLI$ with the regimentation $R \rightarrow \Inevitably\future\neg R$. 
      }

    \item If it wasn't that cold before, it might still be that cold at some point.
	      % \answer{ NAME % collaborator names
	      % }{% body of the argument
	      %   Your answer goes here
	      % }

    \item Either it has rained or it will snow.
	      % \answer{ NAME % collaborator names
	      % }{% body of the argument
	      %   Your answer goes here
	      % }

    \item If it will rain, then it has always been so.
	      % \answer{ NAME % collaborator names
	      % }{% body of the argument
	      %   Your answer goes here
	      % }

    \item If it will always rain, then it must have rained before.
	      % \answer{ NAME % collaborator names
	      % }{% body of the argument
	      %   Your answer goes here
	      % }

    \item It has always been true that it either will rain or it won't.
	      % \answer{ NAME % collaborator names
	      % }{% body of the argument
	      %   Your answer goes here
	      % }

    \item If it has always rained, then it will always have been that it rained before.
	      % \answer{ NAME % collaborator names
	      % }{% body of the argument
	      %   Your answer goes here
	      % }

    \item If it has always rained, then it has always been that it has rained.
	      % \answer{ NAME % collaborator names
	      % }{% body of the argument
	      %   Your answer goes here
	      % }

    \item If it will always rain, then it cannot have always not rained.
	      % \answer{ NAME % collaborator names
	      % }{% body of the argument
	      %   Your answer goes here
	      % }

    \item If has rained and snowed, then it could tomorrow.
	      % \answer{ NAME % collaborator names
	      % }{% body of the argument
	      %   Your answer goes here
	      % }

    \item If rain has always implied clouds, it will always be cloudy if it always rains.
	      % \answer{ NAME % collaborator names
	      % }{% body of the argument
	      %   Your answer goes here
	      % }

    \item If rain lead to snow before, then snow might lead to rain.
	      % \answer{ NAME % collaborator names
	      % }{% body of the argument
	      %   Your answer goes here
	      % }

\end{enumerate}





\section{Temporal Frame Constraints}

\begin{enumerate}

	\item[\bf Relations:] Evaluate the following, providing a proof or counterexample:

    \item Every asymmetric frame is irreflexive.
	      % \answer{ NAME % collaborator names
	      % }{% body of the argument
	      %   Your answer goes here
	      % }

    \item Every irreflexive transitive frame is asymmetric.
	      % \answer{ NAME % collaborator names
	      % }{% body of the argument
	      %   Your answer goes here
	      % }

    \item Every frame that is not irreflexive has neither beginning nor end.
	      % \answer{ NAME % collaborator names
	      % }{% body of the argument
	      %   Your answer goes here
	      % }

    \item Every left and right linear frame is total.
	      % \answer{ NAME % collaborator names
	      % }{% body of the argument
	      %   Your answer goes here
	      % }

    \item Every total frame is left and right linear.
      \answer{ Ben % Collaborators
      % Reviewers
      }{% body of the argument
        Let $\F = \tuple{T, <}$ be a total frame where both $y < x$ and $z < x$ for arbitrary $x, y, z \in T$. 
        By \textsc{tot}, either $y < z$, $y = z$, or $y > z$. 
        Since $x, y, z \in T$ were arbitrary, we may conclude that $\F$ is left linear. 

        Assuming instead that $y > x$ and $z > x$ for arbitrary $x, y, z \in T$, either $y < z$, $y = z$, or $y > z$ follows by \textsc{tot}. 
        Generalizing on $x, y, z \in T$, $\F$ is also right linear. 
        Since $\F$ was an arbitrary total frame, we may conclude that every total frame is left and right linear. 
        \qed
      }

    \item Every frame that is left and right linear is transitive.
	      % \answer{ NAME % collaborator names
	      % }{% body of the argument
	      %   Your answer goes here
	      % }

    \item Every frame that is not right linear is right discrete.
	      % \answer{ NAME % collaborator names
	      % }{% body of the argument
	      %   Your answer goes here
	      % }

    \item Every frame that is dense is both left and right linear.
	      % \answer{ NAME % collaborator names
	      % }{% body of the argument
	      %   Your answer goes here
	      % }

    \item Every frame that is asymmetric and left linear is transitive.
	      % \answer{ NAME % collaborator names
	      % }{% body of the argument
	      %   Your answer goes here
	      % }

    \item There is a dense frame with both a beginning and end.
      \answer{ Ben % Collaborators
      % Reviewers
      }{% body of the argument
        Consider the frame $\F = \tuple{[0,1], <}$ where $[0,1] \subseteq \mathbb{Q}$ and $<$ is the standard ordering of rational numbers.
        Thus for all $i \in (0,1)$, we have:
        \begin{center}
          \begin{tikzpicture}
            % Define the worlds
            \node[world] (0) at (0,0) {$0$};
            \node        (a) at (1.5,0) {\ldots};
            \node[world] (i) at (2.25,0) {$i$};
            \node        (b) at (3,0) {\ldots};
            \node[world] (1) at (4.5,0) {$1$};
            
            % Draw arrows
            \draw[arrow] (0) to (a);
            % \draw[arrow] (a) to (i);
            % \draw[arrow] (i) to (b);
            \draw[arrow] (b) to (1);
          \end{tikzpicture}
        \end{center}
        Since $0 < i < 1$ for all $i \in (0,1)$, it follows that $\F$ has both a beginning and end (it is bounded below and above) and so neither \textsc{inf} or \textsc{inp} hold.

        Given any $x, z \in [0,1]$ where $x < z$, we may let $y = x + \frac{z-x}{2}$ where this is the rational number between $x$ and $z$, and so $x < y < z$.
        Since $x, z \in [0, 1]$ were arbitrary, $\F$ satisfies \textsc{den} as desired.
        \qedp
      }

    \item The relational image of a frame with a beginning and end is finite.
	      % \answer{ NAME % collaborator names
	      % }{% body of the argument
	      %   Your answer goes here
	      % }

    \item The relational image of an asymmetric frame is not a partition.
	      % \answer{ NAME % collaborator names
	      % }{% body of the argument
	      %   Your answer goes here
	      % }

\end{enumerate}




\section{Characterization}

\begin{enumerate}
	\item[\bf Countermodels:] Evaluate the following, providing a proof or $\TL$ countermodel.
    If there is a countermodel, strengthen $\MLmodels[]$ by imposing the weakest set of constraints $C$ which make that claim valid. 
    (You do not need to prove that it is the weakest set of constraints.)

    \item $\MLmodels[] \Past(\metaA \rightarrow \metaB) \rightarrow (\Past\metaA \rightarrow \Past\metaB)$.
	      % \answer{ NAME % collaborator names
	      % }{% body of the argument
	      %   Your answer goes here
	      % }

    \item $\MLmodels[] \past\top$.
	      % \answer{ NAME % collaborator names
	      % }{% body of the argument
	      %   Your answer goes here
	      % }

    \item $\MLmodels[] \metaA \rightarrow \Future \past \metaA$.
	      % \answer{ NAME % collaborator names
	      % }{% body of the argument
	      %   Your answer goes here
	      % }

    \item $\MLmodels[] \Past\Past\metaA \rightarrow \Past\metaA$.
	      % \answer{ NAME % collaborator names
	      % }{% body of the argument
	      %   Your answer goes here
	      % }

    \item $\MLmodels[] \Past\bot \vee \past\Past\bot$.
	      % \answer{ NAME % collaborator names
	      % }{% body of the argument
	      %   Your answer goes here
	      % }

    \item $\MLmodels[] \Past\Future\metaA \rightarrow \always\metaA$.
	      % \answer{ NAME % collaborator names
	      % }{% body of the argument
	      %   Your answer goes here
	      % }

    \item $\MLmodels[] \Past\metaA \rightarrow \Past\Past\metaA$.
	      % \answer{ NAME % collaborator names
	      % }{% body of the argument
	      %   Your answer goes here
	      % }

    \item $\MLmodels[] (\past\top \wedge \metaA \wedge \Future\metaA) \rightarrow \past\Future\metaA$.
	      % \answer{ NAME % collaborator names
	      % }{% body of the argument
	      %   Your answer goes here
	      % }


    \item $\MLmodels[] \Future(\metaA \rightarrow \metaB) \rightarrow (\Future\metaA \rightarrow \Future\metaB)$.
	      % \answer{ NAME % collaborator names
	      % }{% body of the argument
	      %   Your answer goes here
	      % }

    \item $\MLmodels[] \future\top$.
	      % \answer{ NAME % collaborator names
	      % }{% body of the argument
	      %   Your answer goes here
	      % }

    \item $\MLmodels[] \metaA \rightarrow \Past \future \metaA$.
	      % \answer{ NAME % collaborator names
	      % }{% body of the argument
	      %   Your answer goes here
	      % }

    \item $\MLmodels[] \Future\Future\metaA \rightarrow \Future\metaA$.
      \answer{ Ben % Collaborators
      % Reviewers
      }{% body of the argument
        Consider an $\TL$ model $\M = \tuple{T, <, \I}$ where $T = \set{w, u}$, only $w < u$, and $\I(p_1) = \varnothing$ (the interpretation of all other sentence letters is arbitrary):
        \begin{center}
          \begin{tikzpicture}
            % Define the worlds
            \node[world] (w) at (0,0) {$w$};
            \node[below] at (0,-0.5) {$\Future\Future p_1,\ \neg\Future p_1$};
            \node[world] (u) at (3,0) {$u$};
            \node[below] at (3,-0.5) {$\neg p_1,\ \Future p_1$};
            
            % Draw arrows
            \draw[arrow] (w) to (u);
          \end{tikzpicture}
        \end{center}
        Vacuously, every $v \in T$ where $u < v$ is such that $v \in \I(p_1)$, and so $\M, v \MLmodels[] p_1$. 
        Thus $\M, u \MLmodels[] \Future p_1$ by the semantics for $\Future$, and so $\M, w \MLmodels[] \Future\Future p_1$ since $u$ is the only element of $T$ where $w < u$. 
        At the same time, $u \notin \I(p_1)$, and so $\M, u \nMLmodels[] p_1$.
        Since $w < u$, it follows that $\M, w \nMLmodels[] \Future p_1$ by the semantics for $\Future$.
        Thus $\M, w \nMLmodels[] \Future\Future p_1 \rightarrow \Future p_1$ by the semantics for $\rightarrow$. 

        Nevertheless, we may show that $\nMLmodels[\textsc{den}] \Future\Future\metaA \rightarrow \Future\metaA$ by assuming for contradiction that there is an $\TL$ model $\M = \tuple{T, <, \I}$ that satisfies \textsc{den} where $\M, w \nMLmodels[\textsc{den}] \Future\Future\metaA \rightarrow \Future\metaA$ for some $w \in T$. 
        It follows by the semantics for $\rightarrow$ that both: (1) $\M, w \MLmodels[\textsc{den}] \Future\Future\metaA$; and (2) $\M, w \nMLmodels[\textsc{den}] \Future\metaA$.
        It follows from the latter that $\M, u \nMLmodels[\textsc{den}] \metaA$ for some $u \in T$ where $w < u$. 
        Since $\M$ satisfies \textsc{den}, there is some $v \in T$ where $w < v < u$, and so we have:
        \begin{center}
          \begin{tikzpicture}
            % Define the worlds
            \node[world] (w) at (0,0) {$w$};
            \node[below] at (0,-0.5) {$\Future\Future\metaA$};
            \node[world] (v) at (3,-1) {$v$};
            \node[below] at (3,-1.5) {$\Future\metaA$};
            \node[world] (u) at (6,0) {$u$};
            \node[below] at (6,-0.5) {$\neg\metaA,\ \metaA$};
            
            % Draw arrows
            \draw[arrow] (w) to (v);
            \draw[arrow] (w) to (u);
            \draw[arrow] (v) to (u);
          \end{tikzpicture}
        \end{center}
        Since $w < v$, it follows from (1) that $\M, v \MLmodels[\textsc{den}] \Future\metaA$, and so $\M, u\MLmodels[\textsc{den}] \metaA$, contradicting the above.
        Thus $\MLmodels[\textsc{den}] \Future\Future\metaA \rightarrow \Future\metaA$ as desired. 
        \qed
      }

    \item $\MLmodels[] \Future\bot \vee \future\Future\bot$.
	      % \answer{ NAME % collaborator names
	      % }{% body of the argument
	      %   Your answer goes here
	      % }

    \item $\MLmodels[] \Future\Past\metaA \rightarrow \always\metaA$.
	      % \answer{ NAME % collaborator names
	      % }{% body of the argument
	      %   Your answer goes here
	      % }

    \item $\MLmodels[] \Future\metaA \rightarrow \Future\Future\metaA$.
	      % \answer{ NAME % collaborator names
	      % }{% body of the argument
	      %   Your answer goes here
	      % }

\end{enumerate}


\section{Indeterminacy}

\begin{enumerate}

  \item[\bf Evaluate:] Without imposing any restriction on the models of $\TL_\square$, evaluate the following where $p_i \in \SL$, providing a proof or countermodel:

    \item $\MLmodels[] p_i \rightarrow \inevitably p_i$.
	      % \answer{ NAME % collaborator names
	      % }{% body of the argument
	      %   Your answer goes here
	      % }

    \item $\MLmodels[] \metaA \rightarrow \inevitably\metaA$.
	      % \answer{ NAME % collaborator names
	      % }{% body of the argument
	      %   Your answer goes here
	      % }

    \item $\MLmodels[] \past\metaA \vee \past\neg\metaA$.
	      % \answer{ NAME % collaborator names
	      % }{% body of the argument
	      %   Your answer goes here
	      % }

    \item $\MLmodels[] \metaA \rightarrow \Future\past\metaA$.
	      % \answer{ NAME % collaborator names
	      % }{% body of the argument
	      %   Your answer goes here
	      % }

    \item $\MLmodels[] \Past\Future\metaA \rightarrow \always\metaA$.
	      % \answer{ NAME % collaborator names
	      % }{% body of the argument
	      %   Your answer goes here
	      % }

    \item $\MLmodels[] \metaA \rightarrow \Inevitably\metaA$.
	      % \answer{ NAME % collaborator names
	      % }{% body of the argument
	      %   Your answer goes here
	      % }

    \item $\MLmodels[] \future\metaA \vee \future\neg\metaA$.
      \answer{ Ben % Collaborators
      % Reviewers
      }{% body of the argument
        Consider a minimal model $\M = \tuple{T, <, \I}$ for $\TLI$ where $T = {x}$ has just one time, $x \nless x$, and $\I$ is arbitrary. 
        Letting $\T_i = \tuple{T, <}$, we may observe that $\M, \T_i, x \nMLmodels[] \future\metaA$ since there is no $y \in T_i$ where $x < y$ and $\M, \T_i, y \nMLmodels[] \metaA$, and so $\M, \T_i, y \MLmodels[] \neg\metaA$ by the semantics for negation.
        Moreover, $\M, \T_i, x \nMLmodels[] \future\neg\metaA$ since neither is there a $y \in T_i$ where $x < y$ and $\M, \T_i, y \nMLmodels[] \neg\metaA$. 
        It follows that $\M, \T_i, x \nMLmodels[] \neg\future\metaA \rightarrow \future\neg\metaA$ by the semantics for $\rightarrow$, and so $\nMLmodels[] \neg\future\metaA \rightarrow \future\neg\metaA$ by the definition of logical consequence.
        Thus $\nMLmodels[] \future\metaA \vee \future\neg\metaA$ by abbreviation.
        \qedp
      }

    \item $\MLmodels[] \metaA \rightarrow \Past \future \metaA$.
	      % \answer{ NAME % collaborator names
	      % }{% body of the argument
	      %   Your answer goes here
	      % }

    \item $\MLmodels[] \Future\Past\metaA \rightarrow \always\metaA$.
	      % \answer{ NAME % collaborator names
	      % }{% body of the argument
	      %   Your answer goes here
	      % }

\end{enumerate}

%%% Bibliography %%%

% \vfill
% \begin{small} %%Makes bib small text size
%   \singlespacing %%Makes single spaced
%   \bibliographystyle{../../assets/bib_style} %%bib style found locally or in textmf/bibtex/bst
%   \setlength{\bibsep}{0.5pt} %%Changes spacing between bib entries
%   \bibliography{../../assets/modal_history} %%bib database found locally or in textmf/bibtex/bib
%   \thispagestyle{empty} %%Removes page numbers
% \end{small} %%End makes bib small text size

\end{document}
