\documentclass[a4paper, 11pt]{article}                  % paper and font size
\usepackage[top=1.1in, bottom=1.1in]{geometry}          % margins
\usepackage[protrusion=true,expansion=true]{microtype}  % better typography
\usepackage{../assets/problem_set}                   % imports style file
\usepackage{../assets/notation}                   % imports style file

% Set the colors for answer boxes in this document (default is olive green)
% \setanswerboxcolors{Gray!10}{Gray!80}             % For gray theme
% \setanswerboxcolors{blue!5}{blue!40}            % Blue theme
% \setanswerboxcolors{RawSienna!15}{RawSienna!50} % Orange theme  
% \setanswerboxcolors{Plum!10}{Purple!50}         % Purple theme
% \setanswerboxcolors{Green!10}{Green!50}         % Green theme

%----------------------------------------------------------------------------------------
%	TITLE
%----------------------------------------------------------------------------------------

\title{The Modern History of Modal Logic}  % Title
\pset{Problem Set 03: Due March 3rd}              % Problem Set
\date{\today}                              % Date

%----------------------------------------------------------------------------------------

\begin{document}
\maketitle              % Print the title section
\thispagestyle{empty}   % Drop header and page number from the first page

%----------------------------------------------------------------------------------------

%%%%%%%%%%%%%%%%%%%%
%%% INSTRUCTIONS %%%
%%%%%%%%%%%%%%%%%%%%

% 1. Make sure to pull changes (sync) before starting to work on any problems
% 2. Add your name to the problems you want to work on, pushing changes immediately to "check out" the problems
% 3. If a problem has been checked out once, make a copy of the solution block below to contribute your solution below
% 4. If there are already two people working on a problem, try to find another problem to work on
% 5. If you make some progress (even partial) and get stuck, push changes for others to see/help/etc.
% 6. If you have questions or comments, please open a new issue (it's possible to link line numbers) 
% 7. It is best practice to have one sentence per numbered line, making it easier to add comments
% 8. If possibly, try to avoid breaking formal expressions across lines to improve readability
% 9. I will comment on your solutions, adding my name to the reviewers (feel free to add comments/questions as well)
% 10. If the comments have been addressed or no longer relevant, feel free to remove comments
% 11. If you want to add comments, make sure to create a new line below to do so (this will help avoid conflicts)
% 12. All the defined commands can be found in assets/notation.sty but feel free to use what you like
% 13. I will try to include comments alongside the definition of environments in notation.sty to explain their use
% 14. Make sure that the document builds without errors before pushing changes
% 15. Please try to choose a diversity of problem types to work on
% 16. To streamline your proofs, try to unpack claims with an existential flavor before those with a universal flavor
% 17. In using reductio proofs, think about what the easiest thing to negate would be given the intended conclusion
% 18. Sometimes a contraposition style proof is just as easy as a reductio, and sometimes direct is the best of all
% 19. Try to neither skip substantial steps nor to make any unforced moves
% 20. Cutting and pasting from your or other's proofs is perfectly fine as long as the end result is good



%%%%%%%%%%%%%%%%%%%%

\section{Truth-Conditions}

\begin{enumerate}

	\item Why are frames required to be nonempty?
   \answer{}{}{% body of the argument
      %   Begin argument...
      % }

    \item $\M, w \MLmodels[] \Diamond\metaA$ \textit{iff} $\M, u \MLmodels \metaA$ for some $u \in W$ such that $R(w, u)$.
      \answer{ Miguel % Collaborators
      [Ben] % Reviewers
      }{% body of the argument
        By the semantics of $\Box$, $\M, w \MLmodels \Box \metaA \textit{ iff } \M, u \MLmodels \metaA \text{ for every } u \in W \text{ s.t. } R(w,u)$. 
        Substituting $\neg \metaA$ for $\metaA$, $\M, w \MLmodels \Box \neg \metaA \textit{ iff } \M, u \MLmodels \neg \metaA \text{ for every } u \in W \text{ s.t. } R(w,u)$. 
        By the semantics of negation, $\M, w \MLmodels \neg \Box \neg \metaA \textit{ iff } \text{ it is not the case that }\M, u \MLmodels \neg \metaA \text{ for every } u \in W \text{ s.t. } R(w,u)$. 
        This is equivalent to saying $\M, w \MLmodels \neg \Box \neg \metaA \textit{ iff } \M, u \nMLmodels \neg \metaA \text{ for some } u \in W \text{ s.t. } R(w,u)$. 
        By the semantics of negation (and simplifying a double negation to no negation), $\M, w \MLmodels \neg \Box \neg \metaA \textit{ iff } \M, u \MLmodels \metaA \text{ for some } u \in W \text{ s.t. } R(w,u)$. 
        By metalinguistic abbreviation, $\M, w \MLmodels \Diamond \metaA \textit{ iff } \M, u \MLmodels \metaA \text{ for some } u \in W \text{ s.t. } R(w,u)$. \qed
        % [Ben] good to avoid breaking formal expressions over lines
      }

    \item $\M, w \MLmodels[] \Box\metaA$ \textit{iff} $(w)_R \subseteq \interpret{\M}{\metaA}$.
      \answer{ Holden % Collaborators
      [Ben] % Reviewers
      }{% body of the argument
        To show that $\M, w \MLmodels[] \Box\metaA$ \textit{iff} $(w)_R \subseteq \interpret{\M}{\metaA}$, we establish the following for an arbitrary $\ML$ model $\M$, world $w\in W$, and $\metaA \in \wfs{\ML}$.
        \setcounter{equation}{0}
        \begin{align}
          \M, w \MLmodels[] \Box\metaA & ~\textit{ iff}~~ \M, u \MLmodels \metaA \text{ for every } u \in W \text{ such that } R(w, u)\\
          &~\textit{ iff}~~ \text{for every } u \in W, R(w, u) \text{ implies } \M, u \MLmodels \metaA\\
          &~\textit{ iff}~~ \text{for every } u \in W, u \in (w)_R \text{ implies } u \in \interpret{\M}{\metaA}\\
          &~\textit{ iff}~~ (w)_R \subseteq \interpret{\M}{\metaA}
        \end{align}
        (1) follows from the semantics for $\Box\metaA$.
        (2) is just a rephrasing of (1) in the metalanguage.
        (3) Follows from applying the definitions of $(w)_R$ and $\interpret{\M}{\metaA}$.
        (4) Follows from the definition of $\subseteq$.\qed
      }

    \item $\M, w \MLmodels[] \Box(\metaA \rightarrow \metaB)$ \textit{iff} $\interpret{\M}{\metaA}^w \subseteq \interpret{\M}{\metaB}$.
      % \answer{ Name % Collaborators
      % % Reviewers
      % }{% body of the argument
      %   Begin argument...
      % }

    \item $\M, w \MLmodels[] \Box(\metaA \leftrightarrow \metaB)$ \textit{iff} $\interpret{\M}{\metaA}^w = \interpret{\M}{\metaB}^w$.
      % \answer{}{}

    \item $\M, w \MLmodels[] \Diamond\metaA$ \textit{iff} $(w)_R \cap \interpret{\M}{\metaA} \neq \varnothing$.
      % \answer{}{}

    % TODO: move above after pset is complete
    \item $\M, w \MLmodels[] \metaA \vee \metaB$ \textit{iff} $\M, w \MLmodels \metaA$ or $\M, w \MLmodels \metaB$.
      % \answer{}{}

    % NOTE: I dropped this one to make room for an addition above without changing numbering throughout
    % \item $\M, w \MLmodels[] \Diamond\Box\metaA$ \textit{iff} there exists $u \in (w)_R$ where $(u)_R \subseteq \interpret{\M}{\metaA}$.
    %   % \answer{}{}

\end{enumerate}





\section{Frames}

\begin{enumerate}

	\item[\bf Relations:] Evaluate the following, providing a proof or countermodel:

    \item Every serial frame is reflexive.
      \answer{ Ben % Collaborators
      % Reviewers
      }{% body of the argument
        \vspace{-.2in}
        \begin{multicols}{2}
          \noindent
          Letting $W = \set{w_i : 0 \leq i \leq 4}$ where $R = \set{\tuple{w_x, w_y} : y = x + 1 ~~ (\bmod\ 5)}$, we may observe that every world sees some world but no world sees itself.
          Thus $\tuple{W, R}$ is serial without being reflexive, providing a countermodel to the stated claim. 

          We may produce infinitely many countermodels by simply varying the cardinality of $\vert{W} > 1$.
          \qedp  % The diamond is because countermodels are existential instead of universal like proofs.

          \columnbreak

          % NOTE: This is more elaborate than it needs to be, making extra work to write and to check.
          % Instead it is better to use as simple of an example as you can find, adding complexity only when needed.
          % I will put a simpler example below that you can adapt.
          \begin{center}
            \begin{tikzpicture}
              % Define the five worlds in a circle
              \node[world] (w0) at (72 :1.75) {$w_0$};
              \node[world] (w1) at (360:1.75) {$w_1$};
              \node[world] (w2) at (288:1.75) {$w_2$};
              \node[world] (w3) at (216:1.75) {$w_3$};
              \node[world] (w4) at (144:1.75) {$w_4$}; 
            
              % Draw arrows clockwise between worlds
              \draw[arrow] (w0) to[bend right] (w1);
              \draw[arrow] (w1) to[bend right] (w2);
              \draw[arrow] (w2) to[bend right] (w3);
              \draw[arrow] (w3) to[bend right] (w4);
              \draw[arrow] (w4) to[bend right] (w0);
            \end{tikzpicture}
          \end{center}
        \end{multicols}

        \noindent
        NOTE: This solution is needlessly complex.
        I use it as an example of what to avoid though it is a perfectly correct countermodel.
        Instead, try to find the simplest countermodels that you can, adding complexity only as needed.

        % NOTE: This example would not work as is, but could easily be adapted to show what the example above shows.
        % \begin{center}
        %   \begin{tikzpicture}
        %     % Define the two worlds
        %     \node[world] (w) at (0,0) {$w$};
        %     \node[world] (u) at (2,0) {$u$};
        %     
        %     % Draw arrows
        %     \draw[arrow] (w) to[loop left] (w);
        %     \draw[arrow] (w) to (u);
        %   \end{tikzpicture}
        % \end{center}

      }

    \item Every serial symmetric frame is reflexive. \answer {Bailey [Ben]}
    % [Ben] although it makes not difference to the pdf, good to stick to the formatting contentions
    % there are various reasons for this but among them is ease since you can just uncomment the block I provide
    % same goes for indenting (VSCodium should auto indent for you)
    {I provide a counterexample. 
    
    Assume the existence of only one world $w\in W$. 
    By seriality, we posit an additional world $u$ such that $R(w, u)$.
    From symmetry, $R(u,w)$. 
    Then we have a frame in which the relation is serial and symmetric, 
    % [Ben] the word 'complete' means various things so best to drop that adjective
    % also, frames are non-empty by definition, so OK to say that it is a frame
    but for which neither the reflexive relation $R(w,w)$ or $R(u,u)$ holds.}
      % \answer{ Name % Collaborators
      % % Reviewers
      % }{% body of the argument
      %   Begin argument...
      % }
      
    \item How many frames are both symmetric and transitive but not reflexive. \answer {Bailey}
    {The question is difficult to answer in its present form.
    
    First, I show that there is at least as many frames as the natural numbers. 
    Let $n$ be any natural number. 
    Then we can imagine a frame with n-many worlds but for which the accessibility relation is empty. 
    The relation on all such worlds is vacuously symmetric and transitive, although it is not reflexive. 
    
    But there are more worlds that meet this condition than this.
    Let there be a frame which contains W such that w has only one element. 
    Now if this element remains invariant, then we can imagine this as a frame with a natural number infinite many worlds that only contain this element. 
    But we can also "diagonalize" this frame by adding an additional world such that it contains only one, but different, element. 
    So there are more than natural number many such worlds. 
    
    We can iterate this procedure further up the set hierarchy. This means that there is no set of all worlds or all models altogether.
    A fortiori, we cannot give a concrete answer to the question posed in the proof. 
    }
      % \answer{}{}

    \item Every frame that is left and right Euclidean is symmetric and transitive.
      \answer{ Holden % Collaborators
      [Ben] % Reviewers
      }{% body of the argument
      Let $\tuple{W, R}$ be a left Euclidean (\textsc{leu}) and right Euclidean (\textsc{reu}) frame.
      
      First, we show that it is symmetric. Assume $R(x, y)$ holds for some $x, y \in W$. We want to show that $R(y, x)$ holds.
      Note that, since we have $R(x, y)$ and $R(x, y)$, we have $R(x, x)$ by \textsc{leu}.
      Now we now have $R(x, y)$ and $R(x, x)$, so we have $R(y, x)$ by \textsc{reu}.

      Now, we show that it is transitive. Assume $R(x, y)$ and $R(y, z)$ hold for some $x, y, z \in W$. We want to show that $R(x, z)$ holds.
      We have already shown that $R$ is symmetric, so we know $R(z, y)$ holds.
      Then, we have $R(x, y)$ and $R(z, y)$, so by \textsc{leu}, we have $R(x, z)$.

      Hence, $R$ is both symmetric and transitive, as desired. \qed
      }

    \item Every frame that is left and right Euclidean is symmetric and transitive.
      \answer {Bailey
      }{
      Let $\tuple{W, R}$ be a left and right Euclidean frame. We show this frame to be both symmetric and transitive. 
      Let there be arbitrary worlds x, y, and z such that $x,y,z \in W$. From LEU, $R(x, y)$  whenever $R(x, z)$ and $R(y,z)$.
      % [Ben] ok to skip to: Let $x, y \in W$ be arbitrary worlds where... 
      From REU, $R(x, y)$ whenever $R(z,x)$ and $R(z,y)$. We show that the whole frame is symmetric by demonstrating $R(y,x)$.
      % [Ben] Better to use definitions than restate them since otherwise things can get cluttered.
      % It can be nice to leave them in the comments for yourself to reference when writing a proof though.
      We show that the whole frame is symmetric by demonstrating $R(y,x)$.
      % [Ben] yes, but only after assuming the antecedent.
      Since we know already that $R(y,z)$ and $R(x,z)$, then by LEU $R(y,x)$ fulfilling symmetry.
      % [Ben] what we know is that $R(x, y)$ since this is the antecedent of symmetry. We don't know anything else. 
      Further, consider that from the premises that we already knew that $R(y,z)$ and $R(z,x)$.
      % [Ben] here you want to start by assuming the antecedent of transitivity for (new) arbitrary x, y, z
      % OK to use the same variables of course. Maybe just say: Now let...
      Thus, by showing that $R(y,x)$, we likewise show that the whole frame is transitive, as showing the transitivity of all other relations in the frame from the premises is immediate.
      }

    \item Every frame that is symmetric and transitive is left and right Euclidean.
      \answer{ David and Sam % Collaborators
      % % Reviewers
      }{% body of the argument
         Let $\tuple{W,R}$ be a symmetric and transitive frame. Let $x, y, z$ be some arbitrary worlds in $W$. Assume that $R(x,y)$. As the frame satisfies \textsc{sym}, we get: $R(y,x)$. Assume $R(y,z)$. By \textsc{tra}, we get: $R(x,z)$. By \textsc{sym}, we get: $R(z,y)$ and $R(z,x)$. This is a transitive and symmetric frame. From this, we also get that the frame is reflexive (by applying \textsc{tra}). We now can show it satisfies \textsc{leu} and \textsc{reu}. Whenever: $R(x,y)$ and $R(x,z)$, it's the case that $R(y,z)$. Whenever: $R(y,x)$ and $R(x,z)$, it's the case that $R(y,z)$. More informally: as our frame is reflexive, symmetric, and transitive, every world `sees' every world. So, it is left and right Euclidean. Generalising on $x, y, z \in W$, we may conclude that $\tuple{W,R}$ is left and right Euclidean.  
         % [Ben] good to separate sentences on their own numbered lines so I can comment by making new lines
         % Otherwise I risk merge conflicts.
         % First sentence is good.
         % Then you want to skip right to assuming the antecedent of one of the constraints for arbitrary worlds, saying which constraint.
         % Then use those assumptions to get to the conclusion.
         % Then repeat for the other constraint.
         % Sometimes less is more since adding extra stuff, even if true, can clutter the flow of the proof.
         % Good to only add what is needed to get from (in this case) antecedent to consequent of each constraint.
         % Use paragraph breaks between parts of the proof.
         % There are reflexive, symmetric, transitive frames where NOT every world sees every other.
      }

    \item Every frame that is transitive and both left and right Euclidean is symmetric.
       \answer{ David and Sam% Collaborators
      % % Reviewers
       }{% body of the argument
         We've seen above that left and right Euclidean frames suffice for symmetry. So, a frame that is Euclidean and transitive will also suffice for symmetry. 
         % [Ben] This proof has a different shape, starting from the antecedent of symmetry and appealing to different constraints to proceed.
       }

    \item Every frame that is symmetric and left Euclidean is transitive.
      \answer{ Ben % Collaborators
      % Reviewers
      }{% body of the argument
        Let $\tuple{W, R}$ be a symmetric (\textsc{sym}) and left Euclidean (\textsc{leu}) frame.
        Since $R$ is transitive (\textsc{tra}) if $R = \varnothing$, we may assume otherwise.
        Letting $R(x, y)$ and $R(y, z)$ for arbitrary $x,y,z \in W$, we aim to show that $R(x, z)$. 

        Since $R(z, y)$ follows by \textsc{sym}, we know $R(x, z)$ by \textsc{leu}.
        Generalizing on $x, y, z \in W$, we may conclude that $\tuple{W, R}$ is transitive as desired. 
        \qed
      }

    \item There is a finite serial transitive frame that is neither reflexive nor symmetric.
       \answer{ David and Sam% Collaborators
      % % Reviewers
       }{% body of the argument
        Let $\tuple{W,R}$ be a finite serial (\textsc{ser}) transitive (\textsc{tra}) frame. By \textsc{ser}, every world seems some world. That is: $\forall x, y \in W$, either $R(x,y)$ or $R(y,x)$. By \textsc{tra}, if $R(x,y), R(y,z)$ then $R(x,z)$. 
        % [Ben] ...sees some world...
        % serial frames: \forall x \exists y R(x,y)
        
        Here's the frame. Let $x, y, z$ be some arbitrary worlds in $W$. Let $R(x,y)$ and $R(y,z)$. By \textsc{sym}, let: $R(z,y)$. By \textsc{tra}, we get: $R(x,z)$. By \textsc{tra} again, we get: $R(z,z)$. By \textsc{tra} again, we get: $R(y,y)$. But: nothing sees $x$. So, the model is not symmetric: even though $R(x,y)$, it is not the case that $R(y,x)$. It is also not reflexive. Even though $x \in W$, it is not the case that $R(x,x)$. 
        % [Ben] this problem is a little subtle
        % The best way to proceed would be to think about the simplest non-reflexive frame and, separately, to the simplest non-symmetry frame.
        % Then find the simplest frame that is neither reflexive nor symmetric.
        % Then think about how to make these serial without making them reflexive or symmetric.
       }

    \item A symmetric frame is left Euclidean just in case it is right Euclidean.
      \answer{ Sam % Collaborators
      [Ben] % Reviewers
      }{I first show that a symmetric right Euclidean frame is left Euclidean, and then I show that a symmetric left Euclidean frame is right Euclidean. 

      To show that a symmetric right Euclidean frame is left Euclidean, let $\tuple{W,R}$ be a symmetric (\textsc{sym}) and right Euclidean (\textsc{reu}) frame. Suppose that there are 3 worlds, x, y, z. Moreover, suppose that $R(x, y)$ and $R(x, z)$. 
      % [Ben] This is minor, but we don't need any cardinality assumptions (it could be x = y = z).
      % So better wording is: Assume $R(x, y)$ and $R(x, z)$ for arbitrary $x, y, z \in W$.
      
      From \textsc{sym}, we get that $R(y, x)$ and $R(z, x)$. Given that $R(y, x)$ and $R(z, x)$, if the frame is left Euclidean, then we need to show $R(y, z)$ and $R(z, y)$. 
      % [Ben] The first move is good, and then all you need to do is use \textsc{reu}.
      % A single application gets you what you need to conclude \textsc{leu} (since x, y, and z were arbitrary).
      % What we need to show is just $R(y, z)$ since the other comes for free by permuting the order of variables 
      
      From \textsc{reu}, $R(x, y)$, and $R(x, z)$, we get that $R(y, z)$. From \textsc{sym}, we get $R(z,y)$, so the conditions of being left Euclidean are satisfied. 
      % [Ben] Yes exactly. All that is missing here is to clear out the unnecessary steps, making only necessary moves to improve readability.
      
      To show that a symmetric left Eucliean frame is right Euclidean, start with the same model, but let $\tuple{W,R}$ be a symmetric and left Euclidean (\textsc{leu}) frame. Since $R(x, y)$ and $R(x, z)$, if the frame is right Euclidean, then we need to show that $R(y, z)$ and $R(z, y)$. 
      % [Ben] The language of 'the same model' is slightly misleading.
      % We are working with frames (models have interpretations) and we are reasoning about an arbitrary one rather than a particular.
      
      From \textsc{sym}, we know $R(y,x)$ and $R(z, x)$. Given that and \textsc{leu}, we know that $R(y, z)$. Given \textsc{sym} we also have $R(z, y)$, so the conditions of being right Euclidean are satisfied. 
      % [Ben] This proof has all the right ideas, it just needs to be boiled down to essentials.
      }

    \item The relational image of a transitive, symmetric, reflexive frame is a partition.
       \answer{ David and Sam % Collaborators
      [Ben] % Reviewers
       }{% body of the argument
       Let $\tuple{W,R}$ be a transitive, symmetric, reflexive frame. 
       In order for it to be a partition, it must exclude the empty set, each subset ought to be disjoint, and it needs to be exhaustive (i.e., include all worlds $\in W$). Let's go one by one. 

       Let $[x]$ be $\{y \in W | Rxy\}$. Call this the `accessibility set of $x$'. We'll show that, via the accessibility sets of each member, we form a partition. That is: the accessibility set of each member provides a set of subsets that satisfy the conditions for partition. 
       % [Ben] The notation [x] you introduce is very natural (assuming R to be given), but we already have notation for this in the Logic Notes: (x)_R.
       % The approach is otherwise perfect
      
       By \textsc{ref}, we know that the empty set is excluded. $\forall w \in W (Rww)$. So, by the accessibility set, we exclude the empty set. 
       % [Ben] Yes, but might help to clarify that every world sees itself, so sees something, so every accessibility set is nonempty, so the empty set is excluded.

       By \textsc{ref}, we also know that every world is accounted for (i.e., satisfy covering). As $\forall w \in W (Rww)$, $\forall w \in W, w \in [w]$. 
       % [Ben] Could help to clarify that since w \in (w)_R where (w)_R is in \img(R) for all w \in W, we know that for every w \in W there is some Y in \img(R) where w \in Y.

       To show that there are no disjoint members, we can show that $\forall w, v \in W$ if $[w] \cap [v]$ then $[w] = [v]$. Two sets are disjoint just in case they include some (but not all) of the same members. So, to show they're not disjoint, we can show that: if they intersect (i.e., share a member), then they're the same set. 
       % [Ben] Rather: To show that all distinct elements of \img(R) are disjoint...
       % Close. You want: if (w)_R \cap (v)_R \neq \varnothing, then...
       % The approach is perfect

       Suppose that: $u \in ([w] \cap [v])$. That is: $Rwu$ and $Rvu$. By \textsc{sym}, we get: $Ruv$. 
       From $Rwu$ and $Ruv$, we get $Rwv$, by \textsc{tra}. We can then show that $\forall x \in [v], Rwx$. If $x \in [v]$, then $Rvx$. From $Rwv$ and $Rvx$, by \textsc{tra} we get $Rwx$. So: $[v] \subseteq [w]$. 
       % [Ben] Good. It would improve clarity to drop the second sentence, starting the third with: Let $x \in (v)_R$ be arbitrary...

       We can do the same thing the other way round. Suppose again that: $u \in ([w] \cap [v])$. That is: $Rwu$ and $Rvu$. By \textsc{sym}, we get: $Ruw$. 
       From $Rvu$ and $Ruw$, we get $Rvw$, by \textsc{tra}. We can then show that $\forall x \in [w], Rvx$. After all: $\forall x$ s.t. $Rvx$, by \textsc{tra}, we get $Rvx$. So: $[v] \subseteq [w]$. But, if $[v] \subseteq [w]$ and $[w] \subseteq [v]$, then $[w] = [v]$. 

       So, if there is any intersection between two accessibility set, then they are the same set. Hence, satisfying the disjoint constraint for partition. 
       % [Ben] There is always an intersection between any two sets.
       % You mean: ...if the intersection between two accessibility sets is nonempty, ...

       The accessibility set of each member of a symmetric, transitive, reflexive frame then satisfies all the constraints on a partition. 

       The definition of relational image was interested in the accessibility set classes, as I defined them here. So, the relational image of a transitive, symmetric, reflexive frame is a partition, as the set of accessibility sets forms a partition. 
       % [Ben] Nice work!
       % Clarity may be improved by dropping the first sentence, and joining with the paragraph above.
       }

    \item Every total frame is a partition.
      \answer{ David % Collaborators
      % % Reviewers
      }{% body of the argument
      Let $\tuple{W,R}$ be a total frame that is reflexive. Let $W = {x, y}$.
      % [Ben] Use $\set{ ... }$ or $\{ ... \}$ 
      % Also, we don't want to assume more than we are given, i.e., just that it is a total frame.
      As it satisfies \textsc{ref}, $Rxx$ and $Ryy$. As it's total, stipulate: $Rxy$. Now every world sees some world ($x$ sees $y$ and itself, $y$ sees itself). This does not form a partition as the corresponding set is: $\{\{x,y\}, \{y\}\}$. The members are not disjoint.
      % [Ben] You are right to conclude that *this* total frame is a partition, but what about all the others?
 
      }

    \item There is a symmetric total frame that is not a partition.
       \answer{ David % Collaborators
      % % Reviewers
       }{% body of the argument
       We know that any equivalence relation $R$ over some set partitions that set. 
       So, to show that every symmetric total frame is a partition, we can show that every such frame involves an equivalence relation (i.e., satisfies \textsc{tra}, \textsc{sym}, \textsc{ref}). 
       By \textsc{tot}, we know: $\forall x, y \in W$, either $Rxy$ or $Ryx$. 
       As the frame is symmetric, we know: $\forall x,y \in W$ if $Rxy$, then $Ryx$. 
       By combining \textsc{tot} and \textsc{sym}, we get the following: $\forall x, y \in W, Ryx$ and $Rxy$. \textsc{tot} requires one of these worlds (i.e., either $Rxy$ or $Ryx$) and \textsc{sym} guarantees that the other is true too. 
       % [Ben] \textsc{tot} requires one of the relations to hold (rather than 'one of these worlds')
       This frame is transitive. \textsc{tra} required that: if $Rxy$ and $Ryz$, then $Rxz$. By \textsc{tot}, we know that either $Rxz$ or $Rzx$. By \textsc[sym], we know that both: $Rxz$ and $Rzx$. So, the frame is transitive. 
       We also know that the frame is reflexive. \textsc[tra] says: if $Rxy$ and $Ryz$, then $Rxz$. 
       We know, in our frame that $Rxy$ and $Ryx$, by \textsc{sym}. So, by \textsc{tra}, we know that $Rxx$. The same can be done for any world in the frame. So, the frame is reflexive. 
       We've shown that a symmetric total frame involves an equivalence relations. An equivalence relation over some set partitions that set. So: there is no symmetric total frame that is not a partition. 
       % [Ben] your strategy is good but could be simplified. 
       % assume R(x,y) and $R(y,z)$ to show that $R(x,z)$ and so transitive 
       % by \textsc{tot}, either $R(x,z)$ or $R(z,x)$, where \textsc{sym} guarantees that both hold
       % thus $R(x,z)$ is guaranteed 
       % next observe that \textsc{ref} is immediate from \textsc{tot}
       % so it is an equivalence relation given your assumptions
       % so the claim holds by \#2.11

       }

	\item[\bf Countermodels:] Evaluate the following, providing a proof or countermodel.
    If there is a countermodel, replace $K$ with the weakest set of constraints $C$ to make the claim hold. 
    You do not need to prove that it is the weakest set of constraints.

    \item $\Box \metaA \MLmodels[K] \Box\Box \metaA$
      \answer{ Holden % Collaborators
      [Ben] % Reviewers
      }{% body of the argument
        This does not hold in general.
        Consider a frame where $W=\set{w_1, w_2, w_3}$ and $R=\set{\tuple{w_1,w_2}, \tuple{w_2,w_3}}$ where $\I(p_0)=\set{w_2}$.
        Let $\M = \tuple{W, R, \I}$.
        It follows from the definition of $\MLmodels$ that $\M, w_2 \MLmodels p_0$.
        Note that $(w_1)_R = \set{w_2}$, and $w_2 \in \interpret{\M}{p_0}$, so $(w_1)_R \subseteq \interpret{\M}{p_0}$ and by \textbf{\#1.11}, $\M, w_1 \MLmodels \Box p_0$.
        However, note that $\M, w_3 \not\MLmodels p_0$, so $w_3 \notin \interpret{\M}{p_0}$.
        Also, $(w_2)_R = \set{w_3}$, so $(w_2)_R \not\subseteq \interpret{\M}{p_0}$.
        Hence, by \textbf{\#1.11}, $\M, w_2 \not\MLmodels \Box p_0$.
        So $w_2 \notin \interpret{\M}{\Box p_0}$.
        Therefore, $(w_1)_R \not\subseteq \interpret{\M}{\Box p_0}$, and by \textbf{\#1.11}, $\M, w_1 \not\MLmodels \Box\Box p_0$. \qedp
        
        \medskip
        
        Restricting to \textsc{transitive} (\textsc{tra}) models, we may show $\Box\metaA \MLmodels[\textsc{tra}] \Box\Box\metaA$.
        Let $\M$ be any $\ML$ model that satisfies \textsc{tra}, $\metaA \in \wfs{\ML}$, and $w \in W$ where $\M, w \MLmodels \Box \metaA$.
        Assume towards a contradiction that $\M, w \not\MLmodels \Box\Box \metaA$.
        Then, by \textbf{\#1.11}, $(w)_R \not\subseteq \interpret{\M}{\Box\metaA}$.
        Hence, there is some $u \in (w)_R$ such that $u \notin \interpret{\M}{\Box\metaA}$.
        % [Ben] Note that this same conclusion could be drawn directly from the semantics for \Box
        % 1.11 is sometimes very helpful and sometimes can add an extra step
        So $\M, u \not\MLmodels \Box\metaA$.
        Then, by \textbf{\#1.11}, $(u)_R \not\subseteq \interpret{\M}{\metaA}$.
        Hence, there is some $v \in (u)_R$ such that $v \notin \interpret{\M}{\metaA}$.
        But by \textsc{tra}, since $u \in (w)_R$ and $v \in (u)_R$, we have $R(w, u)$ and $R(u, v)$ and hence $R(w, v)$.
        So, $v \in (w)_R$.
        But then $(w)_R \not\subseteq \interpret{\M}{\metaA}$.
        So by \textbf{\#1.11}, $\M, w \not\MLmodels \Box \metaA$, which is a contradiction.
        So we conclude that $\M, w \MLmodels \Box\Box \metaA$.\qed
      }

    \item $\Box \metaA \MLmodels[K] \metaA$
       \answer{ David, Joaquin % Collaborators
      % % Reviewers
       }{% body of the argument
      Throughout, let $C$ be the proposed set of constraints on $R$ replacing $K$. We first show that the proposed $C$ is sufficient to arrive at the proposed theorem. Then we show that it is necessary; that if $R$ does not satisfy the proposed $C$, then there exists a countermodel to the theorem.

      Define the indegree of a world $w$ as $indeg(w)$, which is the size of $S = {u : w \in (w)_R}$, and definte the outdegree of $w$ as $outdeg(w)$, which is the size of $S = (w)_R$.

      Claim: $C: \forall w, w \in (w)_R$.

      Sufficiency: if $\Box p$ holds at $w$, then $p$ holds at all of $(w)_R$. By $C$, $w \in (w)_R$. Then $p$ holds at $w$. Thus, $\Box p \rightarrow p$, as desired.

      Necessity: Suppose $C$ fails. Then there exists $w$ with $w \notin (w)_R$. Assign $p$ to be true in all of $(w)_R$, and false in $w$. By construction, this is well-defined. Then $\Box p$, by construction, yet also $\neg p$. So the theorem fails, as desired.
       }

    \item $\metaA \MLmodels[K] \Box \metaA$
       \answer{ David, Joaquin % Collaborators
      % % Reviewers
       }{% body of the argument
      Claim: $C: \forall w, (w)_R \subseteq {w}$.

      Sufficiency: Assume $p$ at $w$. Then $p$ holds at all elements of ${w}$, trivially. Then $p$ holds over all subsets of ${w}$. Then by $C$, $p$ holds at all of $(w)_R$. Then $\Box p$ at $w$. Thus $p \rightarrow \Box p$, as desired.

      Necessity: Suppose $C$ is violated. Then there exists $w$ with $(w)_R \not\subseteq {w}$. This is equivalent to $ \exists u \in (w)_R : u \not\in {w}$. This is equivalent to $ \exists u \in (w)_R: u \not= w$.
      Let $p$ be true at $w$, false at $u$ described above. By construction, as $u \not= w$, this is well-defined. Then $p$ holds at $w$. Yet $\exists u \in (w)_R$ with $\neg p$ at $u$.
      Then $\neg\Box p$ at $w$. So the theorem fails, as desired.
       }

    \item $\Box(\metaA \vee \metaB) \MLmodels[K] \Box \metaA \vee \Box \metaB$
      \answer{ Helena % Collaborators
      % Reviewers
      }{
        Let $W = \set{w_0, w_1, w_2}$ and $R = \set{\tuple{w_0, w_1}, \tuple{w_0,w_2}}$, where $\I(p_1)  = \set{w_1}$, $\I(p_2) = \set{w_2}$. 
        Let $\M=\tuple{W, R, \I}$. It follows from the definition of $\MLmodels$ that $\M, w_1 \MLmodels p_1$ and $\M, w_2 \MLmodels p_2$. Therefore, $w_1 \in \interpret{\M}{p_1}$ and $w_2 \in \interpret{\M}{p_2}$. 
        Moreover, given \textbf{\#1.4}, $\interpret{\M}{p_1} \subseteq \interpret{\M}{p_1} \cup \interpret{\M}{p_2} = \interpret{\M}{p_1 \lor p_2}$. 
        It follows that $w_1 \in \interpret{\M}{p_1 \lor p_2}$. Similarly, $w_2 \in \interpret{\M}{p_1 \lor p_2}$. 
        Note that $(w_0)_R = \set{w_1,w_2}$. So $(w_0)_R \subseteq \interpret{\M}{p_1 \lor p_2}$, and by \textbf{\#1.11}, $\M, w_0 \MLmodels \Box(p_1 \lor p_2)$. 

        However, note that $\M, w_1 \not\MLmodels p_2$, so $w_1 \not\in \interpret{\M}{p_2}$. 
        And since $w_1 \in (w_0)_R$, it follows that $(w_0)_R \not\subseteq \interpret{\M}{p_2}$. Hence by \textbf{\#1.11}, $\M, w \not \MLmodels \Box p_1$.
        Similarly, since $\M, w_2 \not\MLmodels p_1$, it follows that $w_2 \not\in \interpret{\M}{p_1}$. 
        And since $w_2 \in (w_0)_R$, it follows that $(w_0)_R \not \subseteq \interpret{\M}{p_1}$. Hence by \textbf{\#1.11}, $\M, w \not \MLmodels \Box p_2$. 
        Therefore, by \textbf{\#1.15}, $\M, w \not \MLmodels \Box p_1 \lor \Box p_2$. 

        \medskip 

        Restricting considerations to the \textsc{right euclidean (reu)} models, we may show that $\Box(\metaA \lor \metaB) \MLmodels[\textsc{reu}] \Box \metaA \lor \Box \metaB$. 
        Let $\M$ be an arbitrary $\ML$ model that satisfies \textsc{reu}, $\metaA, \metaB \in \wfs{\ML}$, and $w \in W$, $\M, w \MLmodels \Box(\metaA \lor \metaB)$. 
        We want to show that for all $w \in W, \M, w \MLmodels[\textsc{reu}] \Box \metaA \lor \Box \metaB$.
        
        Assume for contradiction that there's a world $w_0 \in W$ where $\M, w_0 \not \MLmodels \Box \metaA \lor \Box \metaB$. 
        Given \textbf{\#1.15}, this implies that $\M, w_0 \not \MLmodels \Box \metaA$ and $\M, w_0 \not \MLmodels \Box \metaB$. 
        So by the semantics of necessity, there are worlds $w_1, w_2 \in W$ such that (i) $R(w_0,w_1)$ where $\M, w_1 \not\MLmodels \metaA$ and (ii) $R(w_0,w_2)$ where $\M, w_2 \not\MLmodels \metaB$. 
        Given $R(w_0,w_1)$ and $R(w_0,w_2)$, by \textsc{reu}, it follows that $R(w_1, w_2)$. Since $R(w_1,w_2)$ and $\M,w_2 \not\MLmodels \metaB$, by the semantics of necessity, $\M, w_1 \not \MLmodels \Box \metaB$. 
        Moreover, given $R(w_0, w_1)$ and $R(w_0, w_1)$, by \textsc{reu} it follows that $R(w_1,w_1)$. % i think this is legit tho i'm not sure?
        % [Ben] this is legit!
        And since $\M,w_1 \not\MLmodels \metaA$, by the semantics of necessity, $\M, w_1 \not \MLmodels \Box \metaA$. 
        Therefore, by \textbf{\#1.15}, $\M, w_1 \not \MLmodels \Box \metaA \lor \Box \metaB$. 
        This contradicts our assumption that $\M, w \MLmodels \Box(\metaA \lor \metaB)$ for all $w \in W$. 
        % [Ben] everything is right but this last step doesn't hold
        % think of a model {w, u} where everything sees everything and (\metaA \wedge \neg\metaB) holds in w
        % and (\neg\metaA \wedge \metaB) holds in u
        % then in both worlds, (\metaA \wedge \metaB) is necessary but neither disjunct is
        % since this is an equivalence relation, you might wonder what constraint could be imposed to prevent this
        % you may consider frame constraints beyond those indicated in the notes
        % you are close!
        Therefore, $\Box(\metaA \lor \metaB) \MLmodels[\textsc{reu}] \Box \metaA \lor \Box \metaB$.
        \qed
      }

    \item $\Box \metaA \MLmodels[K] \Diamond \metaA$
      \answer{ Ben % Collaborators
      % Reviewers
      }{% body of the argument
        Letting $W = \set{w}$ and $R = \varnothing$, we may take $\I : \SL \to \wp(W)$ to be an arbitrary interpretation, setting $\M = \tuple{W, R, \I}$.
        Since $(w)_R = \varnothing$ and $\interpret{\M}{p_0} \subseteq W$, we know that $(w)_R \subseteq \interpret{\M}{p_0}$, and so $\M, w \MLmodels[K] \Box p_0$ by \textbf{\#1.11}.

        However, $\M, w \MLmodels[K] \Diamond p_0$ just in case $\M, u \MLmodels[K] p_0$ for some $w \in W$ where $R(w, u)$.
        Since $R = \varnothing$, there is no such $u \in W$, and so $\M, w \nMLmodels[K] \Diamond p_0$.
        Thus we have shown that $\Box \metaA \MLmodels[K] \Diamond \metaA$ does not hold for all $\metaA$.
        \qedp

        \medskip
        Restricting consideration to the \textsc{serial} (\textsc{ser}) models, we may show that $\Box \metaA \MLmodels[D] \Diamond \metaA$.
        Let $\M$ be any $\ML$ model that satisfies \textsc{ser}, $\metaA \in \wfs{\ML}$, and $w \in W$ where $\M, w \MLmodels \Box\metaA$.
        By \textsc{ser}, we know that there is some $u \in W$ where $R(w, u)$, and so $\M, u \MLmodels \metaA$ by the $\ML$ semantics for necessity. 
        Thus $\M, w \MLmodels \Diamond\metaA$ by \textbf{\#1.10}.
        % This originally said 1.11, but I'm fairly sure it was a typo that was supposed to say 1.10, so I changed it -Holden
        Since $\M$ was an arbitrary $\ML$ model that satisfies \textsc{ser} and $w \in W$ was any world, we may conclude that $\Box \metaA \MLmodels[D] \Diamond \metaA$.
        \qed
      }

    \item $\Diamond \metaA \MLmodels[K] \Box \metaA$
      \answer{ Helena % Collaborators
      % Reviewers
      }{
        Let $W = \set{w_0, w_1}$ and $R = \set{\tuple{w_0,w_0},\tuple{w_0, w_1}}$, where $\I(p_0) = \set{w_1}$. 
        Let $\M = \tuple{W, R, \I}$. 
        It follows from the definition of $\MLmodels$ that $\M, w_1 \MLmodels p_0$. 
        Therefore, $w_1 \in \interpret{\M}{p_0}$. 
        Given $w_1 \in (w_0)_R$ and $w_1 \in \interpret{\M}{p_0}$, it follows that $w_1 \in (w_0)_R \cap \interpret{\M}{p_0}$. 
        By \textbf{\#1. 14}, $\M, w_0 \MLmodels \Diamond p_0$. 

        However, note that since $w_0 \not \in \I(p_0)$, it follows from the definition of $\MLmodels$ that $\M, w_0 \not\MLmodels p_0$, so $w_0 \notin \interpret{\M}{p_0}$. 
        And since $w_0 \in (w_0)_R$, it follows that $(w_0)_R \not \subseteq \interpret{\M}{p_0}$.  
        By \textbf{\#1.11}, $\M, w_0 \not \MLmodels \Box p_0$.
        % [Ben] This is perfectly correct, however here it would be easier to use the semantics since \Box is primitive

        \medskip 

        Restricting consideration to the models that satisfy \textsc{reflexive (ref)}, \textsc{symmetric (sym)} and \textsc{transitive (tra)}, we may show that $\Diamond \metaA \MLmodels[S5] \Box \metaA$. 
        Let $\M$ be an arbitrary $\ML$ model that satisfies \textsc{ref}, \textsc{sym} and \textsc{tra}, $\metaA \in \wfs{\ML}$, and $w \in W$ where $\M, w \MLmodels \Diamond \metaA$. 
        We want to show that for all $w \in W, \M, w \MLmodels[S5] \Box \metaA$. 
        
        % actually no idea if this is right
        % [Ben] good catch! requiring R to be an equivalence relation is not strong enough
        % for instance, if \metaA is true in w and false in u where these are the only worlds and they see each other, then \Diamond\metaA is true but \Box\metaA is false (in both worlds)

        Assume for contradiction that there's a world $w_0 \in W$ where $\M, w_0 \not \MLmodels \Box \metaA$. 
        Given the semantics of necessity, there's a world $w_1 \in W$ such that $R(w_0, w_1)$ but $\M, w_1 \not \MLmodels \metaA$ (equivalently $\M, w_1 \MLmodels \lnot \metaA$). 
        However, since $\M, w_0 \MLmodels \Diamond \metaA$, by \textbf{\#1.10}, there's a world $w_2 \in W$ such that $R(w_0, w_2)$ and $\M, w_2 \MLmodels \metaA$. 
        By \textsc{tra}, since $R(w_0, w_1)$ and $R(w_0, w_2)$, it follows that $R(w_1, w_2)$. [TBC]

        % ... and so am stuck here. any hints?
        % also i think there are actually even weaker constraints? but given the resources in the notes I can only think of S5
        % [Ben] see note above. It was a little tricky of me to give problems that require stronger constraints than I provided in the notes...
        % [Joaquin] Note that one interpretation of the proposed theorem
        % is that "if one image of w has phi (diamond phi), then all images
        % of w have phi (box phi)". Under this interpretation, it is
        % natural to think the condition may be that the constraint is
        % "one image of w must be the whole image", and in fact this holds
        % exactly; it is precisely necessary and sufficient.

      }

    \item $\Diamond\Box \metaA \MLmodels[K] \Box\Diamond \metaA$
      \answer{ Helena % Collaborators
      % Reviewers
      }{
        Let $W = \set{w_0, w_1}$ and $R = \set{\tuple{w_0,w_1}}$, where $\I(p_0) = \set{w_1}$. Let $\M = \tuple{W, R, \I}$. 
        It follows from the definition of $\MLmodels$ that $\M, w_1 \MLmodels p_0$, and so $w_1 \in \interpret{\M}{p_0}$.
        Since $(w_1)_R = \varnothing$ and $\interpret{\M}{p_0} \subseteq W$, we know that $(w_1)_R \subseteq \interpret{\M}{p_0}$, and so $\M, w_1 \MLmodels[K] \Box p_0$ by \textbf{\#1.11}. 
        It follows that $w_1 \in \interpret{\M}{\Box p_0}$. 
        Moreover, since $w_1 \in (w_0)_R$, by \textbf{\#1.14}, $\M, w_0 \MLmodels \Diamond \Box p_0$.

        However, note that since $(w_1)_R = \varnothing$, it follows that $(w_1)_R \cap \interpret{\M}{p_0} = \varnothing$, and so by \textbf{\#1.14}, $\M, w_1 \not\MLmodels \Diamond p_0$.
        It follows that $w_1 \not \in \interpret{\M}{\Diamond p_0}$. Then, given $w_1 \in (w_0)_R$, by \textbf{\#1.11} we know $\M, w_0 \not \MLmodels \Box \Diamond p_0$.
        % [Ben] Nice!
        
        \medskip

        Restricting considerations to the \textsc{symmetry (sym)} models, we may show that $\Diamond\Box \metaA \MLmodels[\textsc{sym}] \Box\Diamond \metaA$. 
        Let $\M$ be any $\ML$ model that satisfies \textsc{sym}, $\metaA \in \wfs{\ML}$, and $w \in W$ where $\M, w \MLmodels \Diamond \Box \metaA$. 
        We want to show that for all $w \in W$, $\M, w \MLmodels \Box \Diamond \metaA$.
        % [Ben] since w is arbitrary, we just need to show that it makes the claim true (so can drop the 'for all')
        
        Assume for contradiction that there's world $w_0 \in W$ where $\M, w_0 \not \MLmodels \Box \Diamond \metaA$. 
        Since $\M, w_0 \MLmodels \Diamond \Box \metaA$, there's a world $w_1 \in W$ such that $R(w_0, w_1)$ and $\M, w_1 \MLmodels \Box \metaA$. 
        % [Ben] replace w_0 with the arbitrary world w introduced above
        By \textsc{sym}, we know that $R(w_1,w_0)$. And so by the semantics of necessity, $\M, w_0 \MLmodels \metaA$. 
        Now since $\M, w_0 \not \MLmodels \Box \Diamond \metaA$, by the semantics of necessity, there's a world $w_2 \in W$ where $R(w_0, w_2)$ and $\M, w_2 \not \MLmodels \Diamond \metaA$. 
        By \textbf{\#1.14}, this says that $(w_2)_R \cap \interpret{\M}{\metaA} = \varnothing$. 
        Now note that by \textsc{sym}, since $R(w_0, w_2)$, we know that $R(w_2, w_0)$, hence $w_0 \in (w_2)_R$. 
        But recall that $\M, w_0 \MLmodels \metaA$, and so $w_0 \in \interpret{\M}{\metaA}$. 
        It follows that $w_0 \in (w_2)_R \cap \interpret{\M}{\metaA}$. Contradiction. 
        % [Ben] The essentials here are all correct
        
        Therefore, since $\M$ was an arbitrary $\ML$ model that satisfies \textsc{sym}, we may conclude that $\Diamond\Box \metaA \MLmodels[\textsc{sym}] \Box\Diamond \metaA$.
        \qed
      }

    \item $\Box\Diamond \metaA \MLmodels[K] \Diamond\Box \metaA$
      \answer{ Helena % Collaborators
      % Reviewers
     }{% 
        Let $W = \set{w_0, w_1}$ and $R = \set{\tuple{w_0,w_0}, \tuple{w_0,w_1}, \tuple{w_1,w_0}, \tuple{w_1,w_1}}$, where $\I(p_0) = \set{w_1}$. 
        Let $\M = \tuple{W, R, \I}$. It follows from the definition of $\MLmodels$ that $\M, w_1 \MLmodels p_0$. 
        Therefore, $w_1 \in \interpret{\M}{p_0}$. 
        Given $w_1 \in (w_1)_R$ and $w_1 \in \interpret{\M}{p_0}$, it follows that $w_1 \in (w_1)_R \cap \interpret{\M}{p_0}$. 
        By \textbf{\#1.14}, $\M, w_1 \MLmodels \Diamond p_0$. 
        And since $w_0 \in (w_1)_R$, by the same reasoning we know $\M, w_0 \MLmodels \Diamond p_0$. 
        By the semantics of necessity, $\M, w_0 \MLmodels \Box \Diamond p_0$.

        However, note that since $w_0 \not \in \I(p_0)$, we know that $\M, w_0 \not \MLmodels p_0$. 
        Then, since $R(w_1, w_0)$, by the semantics of necessity, $\M, w_1 \not \MLmodels \Box p_0$. 
        Similarly, since $R(w_0, w_0)$, we know that $\M, w_0 \not \MLmodels \Box p_0$. 
        Then for $(w_0)_R = \set{w_0, w_1}$, $w_0 \not \in \interpret{\M}{\Box p_0}$ and $w_1 \not \in \interpret{\M}{\Box p_0}$. 
        Hence $(w_0)_R \cap \interpret{\M}{ \Diamond p_0} = \varnothing$. 
        By \textbf{\#1.14}, $\M, w_0 \not \MLmodels \Diamond \Box p_0$.

        This is in fact already an S5 frame, so there's no stronger constraint to validate this formula (?).

        % ??? anything went wrong??
        % [Ben] your reasoning here is perfect, though there is a stronger constraint (just not stated in the notes)

        \qed 
        
      }

    \item $\Box \metaA \MLmodels[K] \Box\Diamond \metaA$
     \answer{ Joaquin% Collaborators
     % Reviewers
     }{% body of the argument
     Claim: $C: \forall w, u \in (w)_R, (w)_R \cap (u)_R \not= \emptyset$

     Sufficiency: Suppose $\Box p$ holds at $w$. Then $\forall x \in (w)_R$, $p$ holds at $x$. 
     Assume $A: (w)_R = \emptyset$. Then $\forall x \in (w)_R$, any property whatsoever holds. Then $\forall x \in (w)_R, \Diamond p$. Then $\Box \Diamond p$ at $w$, and we would be done. Now we discharge $A$ and assume that $(w)_R \not= \emptyset$.
     Then let $x \in (w)_R$. By $C$, there exists $y \in (w)_R \cap (x)_R$. Then $y \in (w)_R$. As $\Box p$ at $w$, we have $p$ holds at $y$. Note that $x$ is arbitrarily ranging over $(w)_R$. Then $\forall u \in (w)_R, \exists y \in (u)_R$ such that $p$ holds at $y$. Then $\Box \Diamond p$, as desired.

     Necessity: Suppose $C$ doesn't hold. Then there is some $w$, with some $u \in (w)_R$, such that $(w)_R \cap (u)_R = \emptyset$.
     Assign $p$ to be true at $(w)_R$, $\neg p$ at $(u)_R$. By above, this is well-defined. Then by construction, $\Box p$ holds at $w$. Yet $\forall x \in (u)_R, \neg p$. Then $\neg \Diamond p$ at $u$. Then $\Diamond \neg \Diamond p$ at $w$, by looking at $u$. Then $\neg \Box \Diamond p$. Thus the theorem fails, as desired.
     }

    \item $\MLmodels[K] \neg\Box(\metaA \wedge \neg \metaA)$
      \answer{ Joaquin % Collaborators
      % % Reviewers
      }{% body of the argument
      Claim $C: \forall w, (w)_R \not= \emptyset$.

      Sufficiency: By $C$, $\exists u \in (w)_R$. At $u$, note that $\neg (p \wedge \neg p)$ holds. Then $\Diamond \neg(p \wedge \neg p)$ at $w$, by considering $u$. Then $\neg \Diamond (p \wedge \neg p)$ at $w$, as desired.

      Neccessity: If $C$ is violated, then $\exists w, (w)_R = \emptyset$. Then vacuously, $\forall u \in (w)_R, (p \wedge \neg p)$. Then $\Box (p \wedge \neg p)$ at $w$, so the theorem fails, as desired.
      }

    \item $\Diamond \metaB \MLmodels[K] \neg\Box(\metaA \wedge \neg \metaA)$
      \answer{ Joaquin% Collaborators
      % % Reviewers
      }{% body of the argument
      Claim: $C: \emptyset$.

      Sufficiency: Note that the proof in 23 above holds precisely the same way, with the sole exception that now, $\exists u \in (w)_R$ is justified by $\Diamond q \rightarrow (w)_R \not= \emptyset$, so the above proof holds with no constraints on $C$.

      Necessity: it is impossible to violate $C$, so it is vacuously necessasry.
      }

    \item $\Box \metaA \MLmodels[K] \Box(\metaB \rightarrow \metaA)$.
      \answer{ Joaquin % Collaborators
      % % Reviewers
      }{% body of the argument
      Claim: $C: \emptyset$.

      Sufficiency: Suppose $\Box p$ at $w$. Then $\forall u \in (w)_R, p$. Note that $p \rightarrow (q \rightarrow p)$. Then $p$ holding at $u$ implies that $q \rightarrow p$ holds at $u$. Then $\forall u \in (w)_R, (q \rightarrow p)$. Then $\Box (q \rightarrow p)$ at $w$, as desired.

      Necessity: it is impossible to violate $C$, so it is vacuously necessary.
      }

    \item $\neg\Box \metaA \MLmodels[K] \Box(\metaA \rightarrow \metaB)$.
      \answer{ Joaquin % Collaborators
      % % Reviewers
      }{% body of the argument
      Claim: $C: \forall w, outdeg(w) \leq 1$.

      Sufficiency: suppose $\neg\Box p$ holds at $w$. Then $\Diamond \neg p$ at $w$. Then $\exists u \in (w)_R$ with $\neg p$. By $C$, the size of $(w)_R$ is less than or equal to 1. Then $(w)_R = {u}$. Then $\forall u \in (w)_R, \neg p$. Note that $\neg p \rightarrow (p \rightarrow q)$. Then $\forall u \in (w)_R, (p \rightarrow q)$. Then $\Box (p \rightarrow q)$ at $w$, as desired.

      Neccesity: Suppose $C$ is violated. Then there exists $w$ with outdegree at least 2. Label these $u \not= v, u,v \in (w)_R$. Assign $\neg p$ at $u$, $p \wedge \neg q$ at $v$. As $u \not= v$, this is well-defined. Then $\Diamond \neg p$ at $w$, by considering $u$. Then $\neg \Box p$. Also $\Diamond p \wedge \neg q$, by considering $v$. Then $\Diamond \neg (\neg p \vee q)$. Then $\neg \Box (p \rightarrow q)$. Thus the theorem fails, as desired.  
      }

\end{enumerate}






\section{Characterization}

\begin{enumerate}

	\item[\bf Semantic Proofs:] Provide semantic proofs of the following:

    \item If $\MLmodels[K] \metaA$, then $\MLmodels[K] \Box\metaA$
       \answer{ David % Collaborators
      % % Reviewers
       }{% body of the argument
      %   Begin argument...
       }

    \item $\MLmodels[K] \Box(\metaA \rightarrow \metaB)\rightarrow(\Box\metaA \rightarrow \Box\metaB)$
      \answer{ Juan
      [Ben] % Reviewers
      }{ 
      Let $\M$ be an arbitrary model $\langle W, R, \I \rangle$. Thus $\M$ is a \textit{K}-model. First we show that for any sentences $\metaA$, $\metaB$ of $\L^{\Box}$, any \textit{K}-model, and any $w \in W$, it's not the case that $\M, w \MLmodels \Box(\metaA \rightarrow \metaB)$ and $\M, w \nMLmodels  (\Box \metaA \rightarrow \Box \metaB)$.
      % [Ben] not a big deal but want to alert you to the commands in the notation.tex file (could save trouble)
      % I'm thinking of $\ML$ here 
      
      Suppose for contradiction there's a world $w \in W$ such that $\M, w \MLmodels \Box(\metaA \rightarrow \metaB)$ and $\M, w \nMLmodels (\Box\metaA \rightarrow \metaB)$. By the left conjuct and the semantics for $\Box$ it follows that for every $u \in W$ such that $R(u,w)$, $M, u \MLmodels \metaA \rightarrow \metaB$. By our definition of $\rightarrow$ it in turn follows that $M, u \nMLmodels \metaA$ or $M, u \MLmodels \metaB$.
      Now let's turn to the right conjunct. It says $M, w \nMLmodels (\Box \metaA \rightarrow \Box\metaB)$. By our definition of $\rightarrow$, $\lnot$, and logical consequence it follows that $\M$ is a model where $\M, w \MLmodels \Box \metaA$ and $\M, w \nMLmodels \Box \metaB$.
      % NB: I'm unsure about this step buuuuuuut let's see if it gets us what we want.
      % [Ben] this proof is correct but could be streamlined by focusing on existential claims (possibilities and negated necessities) before universal claims.
      % the idea is to make no unforced moves, unpacking existential before universal
      % The step you refer to here is the one to unpack before all the universal claims since it will give you a particular world which you can then apply the universals to in order to get a contradiction
      
      From this we can derive a contradiction. By our definition of $\Box$ it follows that for every world $u \in W$ such that $R(u, w)$ $\M, u \MLmodels \metaA$. So $\M, u \MLmodels \metaA$. This means the left conjunct can be true only if $\M, u \MLmodels \metaB$. But by our definition of $\lnot$ it follows that $\M, w \MLmodels \lnot \Box \metaB$. By our definition of $\Box$ it follows that there is some $u \in W$ such that $R(u,w)$ and $\M, u \nMLmodels \metaB$. We just showed this isn't so for every $u \in W$ such that $R(u,w)$. Contradiction.
      
      This means that for any $w \in W$ it's not the case that $\M, w \MLmodels \Box(\metaA \rightarrow \metaB)$ and $\M, w \nMLmodels (\Box \metaA \rightarrow \Box \metaB).$ This is equivalent to $\M, w \nMLmodels \Box(\metaA \rightarrow \metaB)$ or $\M, w \MLmodels (\Box \metaA \rightarrow \Box \metaB)$. By our semantics for $\rightarrow$ it follows that $\M, w \MLmodels \Box(\metaA \rightarrow \metaB) \rightarrow (\Box \metaA \rightarrow \metaB)$. Since $\M$ is an arbitrary \textit{K}-model this holds in all \textit{K}-models. By our definition of logical consequence it follows that $\MLmodels[K] \Box(\metaA \rightarrow \metaB) \rightarrow (\Box \metaA \rightarrow \Box \metaB)$. $\qed$.
       }

    \item $\MLmodels[D] \Box \metaA \rightarrow\Diamond \metaA$
       \answer{ David % Collaborators
      % % Reviewers
       }{% body of the argument
      %   Begin argument...
       }

    \item $\MLmodels[T] \Box \metaA \rightarrow \metaA$
      \answer{ Miguel % Collaborators
      [Ben] % Reviewers
      }{% body of the argument
      Let $\M$ be an arbitrary model $\langle W, R, \I \rangle$ such that $R$ is reflexive. Thus $\M$ is a \textit{T}-model. 
      We will first show that for any \textit{T}-model and any sentence $\metaA$ of $\L^{\Box}$ (and world $w \in W$), it is not the case that $\M, w \MLmodels \Box \metaA$ and $\M, w \nMLmodels \metaA$. 
      % [Ben] in case it is easier there are a bunch of commands in notation.tex, e.g., \ML

      Suppose for contradiction that there is a world $w \in W$ such that $\M, w \MLmodels \Box \metaA$ and $\M, w \nMLmodels \metaA$. 
      By the left conjunct and the semantics of $\Box$, for every world $u \in W$ s.t. $R(w,u)$, $\M, u \MLmodels \metaA$. 
      By the right conjunct, $\M, w \nMLmodels \metaA$. Since our frame is reflexive, $R(w,w)$. So for a $u$ s.t. $R(w,u)$ (namely $w$ itself), $\M, u \nMLmodels \metaA$. 
      But this is a contradiction, since we previously established that for every world $u \in W$ s.t. $R(w,u)$, $\M, u \MLmodels \metaA$. 
      So we must negate our assumption, meaning that there is no world $w \in W$ such that $\M, w \MLmodels \Box \metaA$ and $\M, w \nMLmodels \metaA$. 
      
      This is equivalent to saying that for any world $w \in W$ it is not the case that $\M, w \MLmodels \Box \metaA$ and $\M, w \nMLmodels \metaA$. 
      This in turn is equivalent to saying that for any world $w \in W$, $\M, w \nMLmodels \Box \metaA$ or $\M, w \MLmodels \metaA$. 
      By the semantics of $\rightarrow$, for any $w \in W$ $\M, w \nMLmodels \Box \metaA \rightarrow \metaA$. 
      Since $\M$ is any general \textit{T}-model the above statement applies to all \textit{T}-models, so by defintion of logical consequence $\MLmodels[\textit{T}] \Box \metaA \rightarrow \metaA$. \qed
      % [Ben] the proof is correct, though could be streamlined by letting \M be an arbitrary T model where some w is such that the T axiom is false
      % then once you work out a contradiction, it follows immediately that T is valid over all T models
      }

    \item $\MLmodels[B] \metaA \rightarrow \Box\Diamond\metaA$
       \answer{ Juan
      [Ben] % Reviewers
       }{Let $\M$ be an arbitrary model $\langle W, R, \I \rangle$ such that $R$ is symmetric. Thus $\M$ is a \textit{B}-model. First we show that for any sentence $\varphi$ of $L^{\Box}$, any \textit{B-}model, and any $w \in W$, it's not the case that $\M, w \MLmodels \metaA$ and $\M, w \nMLmodels \Box \Diamond \metaA$.
       % [Ben] slightly better to say: assume for contradiction that \M is a B model where \M, w \nMLmodels[B] \metaA \rightarrow \Box\Diamond \metaA for some w \in W. you can then wait until the moment when you need symmetry to call on this fact, e.g., since \M is a B model, R is symmetric, and so... 
       % But these are just style points to ease exposition and readability
       % Also note \ML defined in notation.tex (to save time)
     
       For contradiction suppose $M, w \MLmodels \metaA$ and $M, w \nMLmodels \Box \Diamond \metaA$. The left conjunct says $\M, w \MLmodels \metaA$. The right conjunct says $\M, w \nMLmodels \Box \Diamond \metaA$. By our definition of $\lnot$ it follows from the right conjunct that  $M, w \MLmodels \lnot \Box \Diamond \metaA$. By our definition of $\Box$ it follows that there is some $u \in W$ such that $R(u,w)$ and $M, u \nMLmodels \Diamond \metaA$. By our abbreviation of $\Diamond$ this is equivalent to $M, u \nMLmodels \lnot\Box \lnot \metaA$. By our definition of $\lnot$ (and by simplifying a double negation) it follows that $M, u \MLmodels \Box \lnot \metaA$. And by our definition of $\Box$ it follows that for every $v \in W$ such that $R(v, u)$ $\M, v \MLmodels \lnot \metaA$. $R$ is symmetric. So it follows from $R(u,w)$ that $R(w,u)$. It follows that $M, w \MLmodels \lnot \metaA$. But this contradicts the left conjunct of our initial assumption. Contradiction.
       % [Ben] good to have one sentence per line so I can comment more easily (and more maintainable)
       % The left conjunct can sit (doesn't need to be restated)
       % the right is good to work with before adding a negation (we want to reduce it's syntactic complexity)
       
       This means that for any $w \in W$ it's not the case that $\M, w \MLmodels \metaA$ and $\M, w \nMLmodels \Box \Diamond \metaA$. This is equivalent to $\M, w \nMLmodels \metaA$ or $\M, w \MLmodels \Box \Diamond \metaA$. By our semantics for $\rightarrow$ it follows that $\M, w \MLmodels \metaA \rightarrow \Box \Diamond \metaA$. Since $\M$ is an arbitrary \textit{B}-model this holds for all \textit{B}-models. By our definition of logical consequence $\MLmodels[B] \metaA \rightarrow \Box \Diamond \metaA$. $\qed$ 
       % [Ben] this could be simplified by the setup mentioned above
       % But the proof is perfectly correct
       }

    \item $\MLmodels[4] \Box\metaA \rightarrow \Box\Box\metaA$
      \answer{ Miguel % Collaborators
      [Ben] % Reviewers
      }{% body of the argument
      Let $\M$ be an arbitrary model $\langle W, R, \I \rangle$ such that $R$ is a reflexive and transitive frame. Thus $\M$ is a 4-model. 
      % [Ben] Better to state in reverse, where you can start by assuming for contradiction that there is a 4-model with a world at which the 4 axiom is false.
      % Then once you get a contradiction it is very easy to wrap up.
      We will first show that for any 4-model and any sentence $\metaA$ of $\L^{\Box}$ (and world $w \in W$), it is not the case that $\M, w \MLmodels \Box \metaA$ and $\M, w \nMLmodels \Box \Box \metaA$. 

      Suppose for contradiction that there is a world $w \in W$ such that $\M, w \MLmodels \Box \metaA$ and $\M, w \nMLmodels \Box \Box \metaA$. 
      % [Ben] good to deal with existential (possibilities, negated necessities) claims before universal
      % will help to streamline the proof
      By the left conjunct and the semantics of $\Box$, $\M, u \MLmodels \metaA$ for every $u \in W$ such that $R(w,u)$. 
      Similarly the right conjunct gives $\M, u \nMLmodels \Box \metaA$ for some $u \in W$ such that $R(w,u)$. 
      By the semantics of $\Box$ again, For some $u \in W$ s.t. $R(w,u)$ and for some $v \in W$ s.t. $R(u,v)$, $\M, v \nMLmodels \metaA$. 
      % [Ben] these are the claims to unpack first, applying the universal claims as needed after to these particulars
      Since our frame is transitive and $R(w,u)$ and $R(u,v)$, then $R(w,v)$. 
      Consequently for a $v$ s.t. $R(w,v)$, $\M, v \nMLmodels \metaA$. 
      However, this is a contradiction, since we previously established that $\M, u \MLmodels \metaA$ for every $u \in W$ such that $R(w,u)$. 
      So we must negate our assumption, meaning there is no world $w \in W$ such that $\M, w \MLmodels \Box \metaA$ and $\M, w \MLmodels \Box \metaA$. 
      % [Ben] typo in last line
      
      This means for any world $w \in W$, it is not the case that $\M, w \MLmodels \Box \metaA$ and $\M, w \nMLmodels \Box \Box \metaA$. 
      This is equivalent to saying that for any $w \in W$, $\M, w \nMLmodels \Box \metaA$ or $\M, w \MLmodels \Box \Box \metaA$. By the semantics of $\rightarrow$, $\M, w \MLmodels \Box \metaA \rightarrow \Box \Box \metaA$. Since our $\M$ is a general 4-model, this is true of all 4-models, meaning by definition of logical consequence $\MLmodels[4] \Box \metaA \rightarrow \Box \Box \metaA$. \qed
      }

    \item $\MLmodels[5] \Diamond\metaA \rightarrow \Box\Diamond\metaA$
       \answer{ Juan
      [Ben] % Reviewers
       }{Let $\M$ be an arbitrary model $\langle W, R, \I \rangle$ such that $R$ is reflexive and right Euclidean. Thus $\M$ is an \textit{S5}-model. First we show that for any sentence $\metaA$ of $L^{\Box}$, any \textit{S5}-model, and any $w \in W$, it's not the case that $\M, w \MLmodels \Diamond \metaA$ and $\M, w \nMLmodels \Box \Diamond \metaA$.
       % [Ben] ditto streamlining strategy mentioned above

       Suppose for contradiction that $M, w \MLmodels \Diamond \metaA$ and $\M, w \nMLmodels \Box \Diamond \metaA$. The left conjunct says $M, w \MLmodels \Diamond \metaA$. By the definition of $\Diamond$ proved above it follows that there is some $u \in W$ such that $R(u,w)$ and $\M, u \MLmodels \metaA$. The right conjunct says $M, w \nMLmodels \Box \Diamond \metaA$. By the definition of $\lnot$ it follows that $\M, w \MLmodels \lnot \Box \Diamond \metaA$. By the definition of $\Box$ it follows that there is some $v \in W$ such that $R(v, w)$ and $\M, v \MLmodels \lnot \Diamond \metaA$. By our abbreviation rules it follows that $\M, v \MLmodels \lnot \lnot \Box \lnot \metaA$. Simplying for double negation that's equivalent to $\M, v \MLmodels \Box \lnot \metaA$. And by our definition of $\Box$ it follows that for every $z \in W$ such that $R(z, v)$ $\M, z \MLmodels \lnot \metaA$.
       % [Ben] good to name the results you cite so others can follow
       % ditto avoid adding negation
       
       $R$ is reflexive and right Euclidean. Reflexivity implies $R(w,w)$, $R(u,u)$, $R(v,v)$, and $R(z,z)$. It follows from $R(u,w)$, $R(v,w)$, and right Euclideanity that $R(u,v)$. So it follows from above that $\M, u \MLmodels \lnot \metaA$. But we saw earlier that the left conjunct of our initial supposition entails that $\M, u \MLmodels \metaA$. Contradiction.
       % [Ben] good to make no unforced moves

       This means that for every world $w \in W$ it's not the case that $\M, w \MLmodels \Diamond \metaA$ and $\M, w \nMLmodels \Box \Diamond \metaA$. This is equivalent to $\M, w \nMLmodels \metaA$ or $\M, w \MLmodels \Box \Diamond \metaA$. From our semantics for $\rightarrow$ it follows that $\M, w \MLmodels \Diamond \metaA \rightarrow \Box \Diamond \metaA$. $\M$ is an arbitrary \textit{S-5} model so this holds for all \textit{S-5} models. By our definition of logical consequence it follows that $\MLmodels[5] \Diamond \metaA \rightarrow \Box \Diamond \metaA$.
       % [Ben] this works!
       \qed
  
       }

	\item[\bf Equivalences:] Provide semantic proofs of the following equivalences:

    \item $\neg\Box \metaA \MLequiv[K] \Diamond\neg \metaA$.
       \answer{Juan
      [Ben] % Reviewers
      }{
      Let $\M$ be a model $\langle W, R, \I \rangle$. Thus $\M$ is a \textit{K}-model. Our aim is to show that for any $w \in W$, any $\metaA$ of $\L^{\Box}$, and any \textit{K}-model the following two claims hold: (i) if $\M, w \MLmodels \lnot \Box \metaA$ then $\M, w \MLmodels \Diamond \lnot \metaA$ and (ii) if $\M, w \MLmodels \Diamond \lnot \metaA$ then $\M, w \MLmodels \lnot \Box \metaA$.

      Let us prove claim (i). Suppose for contradiction that $\M, w \MLmodels \lnot \Box \metaA$ and $\M, w \nMLmodels \Diamond \lnot \metaA$. The left conjunct says $\M, w \MLmodels \lnot \Box \metaA$. By our definition of $\Box$ it follows there is some $u \in W$ such that $R(u,w)$ and $\M, u \nMLmodels \metaA$. The right conjunct says that $\M, w \nMLmodels \Diamond \lnot \metaA$. By our definition of $\lnot$ it follows that $\M, w \MLmodels \lnot \Diamond \lnot \metaA$. By our definition of $\Diamond$ (along with simplifying a double negation) it follows that for all $v \in W$ such that $R(v,w)$ $\M, v \MLmodels \metaA$. This implies $\M, u \MLmodels \metaA$ since $R(u,w)$, \textit{contra} what the left conjunct entails. Contradiction. So for every world $w \in W$ it's not the case that $\M, w \MLmodels \lnot \Box \metaA$ and $\M, w \nMLmodels \Diamond \lnot \metaA$. This equivalent to saying that if $\M, w \MLmodels \lnot \Box \metaA$ then  $\M, w \MLmodels \Diamond \lnot \metaA$.
      % [Ben] this is an example where a direct proof may be less cumbersome than a reductio
      
      Now let us prove claim (ii). Suppose for contradiction that $\M, w \MLmodels \Diamond \lnot \metaA$ and $\M, w \nMLmodels \lnot \Box \metaA$. The left conjunct says $\M, w \MLmodels \Diamond \lnot \metaA$. By our definition of $\Diamond$ it follows that there is some $u \in W$ such that $R(u,w)$ and $\M, u \MLmodels \lnot \metaA$. The right conjunct says $\M, w \nMLmodels \lnot \Box \metaA$. By our definition of $\lnot$ it follows that $\M, w \MLmodels \lnot \lnot \Box \metaA$. Simplifying for double negation $\M, w \MLmodels \Box \metaA$. By our definition of $\Box$ it follows that for every $v \in W$ such that $R(v, w)$ $\M, v \MLmodels \metaA$. It follows that $\M, u \MLmodels \metaA$ since $R(u,w)$, \textit{contra} what the left conjunct entails. Contradiction. So for every $w \in W$ it's not the case that $\M, w \MLmodels \Diamond \lnot \metaA$ and $\M, w \nMLmodels \lnot \Box \metaA$. This is equivalent to saying that if $\M, w \MLmodels \Diamond \lnot \metaA$ then $\M, w \MLmodels \lnot \Box \metaA$.
      
      We have proven claims (i) and (ii). $\M$ is an arbitrary \textit{K}-model so they hold for all \textit{K}-models. By our definition of logical consequence it follows that $\lnot \Box \metaA \MLmodels[K] \Diamond \lnot \metaA$ and that $\Diamond \lnot \metaA \MLmodels[K] \lnot \Box \metaA$. By our definition of logical equivalence it follows that $\lnot \Box \metaA \MLequiv[K] \Diamond \lnot \metaA$. $\qed$
      % [Ben] This works!
      }

    \item $\neg\Diamond \metaA \MLequiv[K] \Box\neg \metaA$.
    \answer{Bailey
    
    }{
    Let $\M$ be an arbitrary model $\langle W, R, \I \rangle$. Since no further condition on $R$ must be specified for $M$ to be a \textit{K}-model, $M$ is a \textit{K}-model. I show that for any proposition $\metaA$ in the language $\L^{\Box}$, any \textit{K}-model, and any $w \in W$ that the following claims are true: (i) if $\neg \Diamond \metaA$, then $\Box\neg \metaA$ and (ii) if $\Box\neg \metaA$, then $\neg \Diamond \metaA$
    
    Let us first prove (i), which is nearly immediate. Suppose for contradiction that $\M, w \MLmodels \neg \Diamond \metaA$ and  $\M, w \nMLmodels \Box \neg \metaA$. From the meaning of $\Box$, logical consequence, and double negation elimination, we may rewrite the right conjunct as $M, w \MLmodels \Diamond \metaA$. Contradiction. 
    
    Now let us prove (ii). Suppose for contradiction that $M, w \MLmodels \Box \neg \metaA$ and $M,w \nMLmodels \neg \Diamond \metaA$. From the meaning of logical consequence and the application of double negation elimination, $M, w \MLmodels \Box \neg \metaA$ and $M,w \MLmodels \Diamond \metaA$. Consider the first conjunct. By our semantics, for every world $v \in W$ s.t. $R(w,v)$ $M, v \MLmodels \neg \metaA$. Now consider the second conjunct. The second conjunct means that there is some world $u$ s.t. $u \in W$, $R(w,u)$ and $M, u \MLmodels \metaA$. Since $u$ is one of the worlds $v$, it follows that $M, u \MLmodels \metaA$ and $M, u \MLmodels \neg \metaA$. Contradiction. 
    
    I generalize strategy in 3.8 to generate a lemma for future use. Let $C$ be any collection of conditions on the accessibility relation $R$. If a statement of biconditional form is true for an arbitrary \textit{C}-model, it is true for all \textit{C}-models. Call the first element in the biconditional $A$ and the second element in the biconditional $B$. Since it is a biconditional, the order is actually arbitrary. If $A \Leftrightarrow B$, it follows from the definition of logical consequence that $A \MLmodels B$ and $B \MLmodels A$. From logical equivalence, it follows that $A \MLequiv[C] B$.
    
    I call this lemma the equivalence lemma. 
    
    By the equivalence lemma, (i), and (ii), $\neg \Diamond \phi \MLequiv[K] \Box \neg \phi$. 

    }

    \item $\Diamond(\metaA \vee \metaB) \MLequiv[K]  \Diamond\metaA \vee \Diamond\metaB$
    \answer {Bailey
    
    }{
      Let us help ourselves to the standard facts about K-models from 3.8 and 3.9. We show the biconditional (i.) if $M,w \MLmodels \Diamond(\metaA \vee \metaB)$, then $M, w \MLmodels \Diamond \metaA \vee \Diamond \metaB.$

      First, let us prove (i). Assume for contradiction that $M, w \MLmodels \Diamond(\metaA \vee \metaB)$ and $M, w \nMLmodels \Diamond \metaA \vee \Diamond \metaB$. By the semantic rules for negation, we can rewrite the right conjunct $M, w \MLmodels \neg(\Diamond \metaA \vee \Diamond \metaB)$. Consider the left conjunct. This means that there is some world $u$ s.t. $R(w,u)$ and that $M,u \MLmodels \metaA \vee \metaB$. Now consider the right conjunct. This means that there is not some world $v$ s.t. $R(w,v)$ where $M,v \MLmodels \metaA$ and not some world $q$ s.t. $R(w,q)$ and $M, q \MLmodels \metaB$. Now let us reconsider the world $u$, which is some world $v$ and some world $q$. We will first note that $\metaA$ cannot be true at $u$ because of the conditions on $v$. By disjunctive syllogism on $\metaA \vee \metaB$ and $\neg \metaA$, we know that $\metaB$ must be true at $u$. But recall that we have stipulated that $\neg \metaB$ must be true at $u$ through our condition on $q$, which stated that there is not some world related by $R$ to $w$ where $\metaB$ is true. Contradiction.

      The proof for (ii) is similar. Suppose $M,w \MLmodels \Diamond \metaA \vee \Diamond \metaB$ and $M, w \nMLmodels \Diamond(\metaA \vee \metaB)$. We rewrite the right conjunction from the semantic rules for negation as $M, w \MLmodels \neg \Diamond(\metaA \vee \metaB)$. We know from the left conjunct that either there is some world $u$ s.t. $R(w,u)$ and $M,u \MLmodels \metaA$ or some world $v$ s.t. $R(w,v)$ and $M,v \MLmodels \metaB$. Now consider the right conjunct, which tells us that there is no world $q$ s.t. $R(w,q)$ and $M, q \MLmodels \metaA \vee \metaB$. Notice that the world $u$ and the world $v$ are some such worlds $q$. Arbitrarily, consider $u$, although we may also consider $v$. The semantics for $\vee$ mean that both $\metaA$ and $\metaB$ must be false at $u$. However, we know by stipulation that $\metaA$ is true at $u$. Contradiction. 

      By the equivalence lemma proved in 3.9, $\Diamond(\metaA \vee \metaB) \MLequiv[K]  \Diamond\metaA \vee \Diamond\metaB$
    }
      % \answer{ Name % Collaborators
      % % Reviewers
      % }{% body of the argument
      %   Begin argument...
      % }

    \item $\Diamond\Box \metaA \MLequiv[B] \Diamond\Box\Diamond\Box\metaA$ 
     \answer {Bailey}
      % \answer{}{}

    \item $\Box\Box \metaA \MLequiv[4] \Box \metaA$.
      % \answer{}{}

    \item $\Box\Diamond \metaA \MLequiv[5] \Diamond \metaA$.
      % \answer{}{}

    \item $\Diamond\Box \metaA \MLequiv[5] \Box \metaA$.
      % \answer{}{}

\end{enumerate}


%%% Bibliography %%%

% \vfill
% \begin{small} %%Makes bib small text size
%   \singlespacing %%Makes single spaced
%   \bibliographystyle{../../assets/bib_style} %%bib style found locally or in textmf/bibtex/bst
%   \setlength{\bibsep}{0.5pt} %%Changes spacing between bib entries
%   \bibliography{../../assets/modal_history} %%bib database found locally or in textmf/bibtex/bib
%   \thispagestyle{empty} %%Removes page numbers
% \end{small} %%End makes bib small text size

\end{document}
