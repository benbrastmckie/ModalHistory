\documentclass[a4paper, 11pt]{article}                  % paper and font size
\usepackage[top=1.1in, bottom=1.1in]{geometry}          % margins
\usepackage[protrusion=true,expansion=true]{microtype}  % better typography
\usepackage{../assets/problem_set}                   % imports style file
\usepackage{../assets/notation}                   % imports style file

% Set the colors for answer boxes in this document (default is olive green)
% \setanswerboxcolors{Gray!10}{Gray!80}             % For gray theme
% \setanswerboxcolors{blue!5}{blue!40}            % Blue theme
% \setanswerboxcolors{RawSienna!15}{RawSienna!50} % Orange theme  
% \setanswerboxcolors{Plum!10}{Purple!50}         % Purple theme
% \setanswerboxcolors{Green!10}{Green!50}         % Green theme

%----------------------------------------------------------------------------------------
%	TITLE
%----------------------------------------------------------------------------------------

\title{The Modern History of Modal Logic}  % Title
\pset{Problem Set 03: Due March 3rd}              % Problem Set
\date{\today}                              % Date

%----------------------------------------------------------------------------------------

\begin{document}
\maketitle              % Print the title section
\thispagestyle{empty}   % Drop header and page number from the first page

%----------------------------------------------------------------------------------------

%%%%%%%%%%%%%%%%%%%%
%%% INSTRUCTIONS %%%
%%%%%%%%%%%%%%%%%%%%

% 1. Make sure to pull changes (sync) before starting to work on any problems
% 2. Add your name to the problems you want to work on, pushing changes immediately to "check out" the problems
% 3. If a problem has been checked out once, make a copy of the solution block below to contribute your solution below
% 4. If there are already two people working on a problem, try to find another problem to work on
% 5. If you make some progress (even partial) and get stuck, push changes for others to see/help/etc.
% 6. If you have questions or comments, please open a new issue (it's possible to link line numbers) 
% 7. It is best practice to have one sentence per numbered line, making it easier to add comments
% 8. If possibly, try to avoid breaking formal expressions across lines to improve readability
% 9. I will comment on your solutions, adding my name to the reviewers (feel free to add comments/questions as well)
% 10. If the comments have been addressed or no longer relevant, feel free to remove comments
% 11. If you want to add comments, make sure to create a new line below to do so (this will help avoid conflicts)
% 12. All the defined commands can be found in assets/notation.sty but feel free to use what you like
% 13. I will try to include comments alongside the definition of environments in notation.sty to explain their use
% 14. Make sure that the document builds without errors before pushing changes
% 15. Please try to choose a diversity of problem types to work on
% 16. To streamline your proofs, try to unpack claims with an existential flavor before those with a universal flavor
% 17. In using reductio proofs, think about what the easiest thing to negate would be given the intended conclusion
% 18. Sometimes a contraposition style proof is just as easy as a reductio, and sometimes direct is the best of all
% 19. Try to neither skip substantial steps nor to make any unforced moves
% 20. Cutting and pasting from your or other's proofs is perfectly fine as long as the end result is good



%%%%%%%%%%%%%%%%%%%%

\section{Truth-Conditions}

\begin{enumerate}

	\item Why are frames required to be nonempty?
	      % \answer{ NAME % collaborator names
	      % }{% body of the argument
	      %   Your answer goes here
	      % }

	\item $\M, w \MLmodels[] \metaA \vee \metaB$ \textit{iff} $\M, w \MLmodels \metaA$ or $\M, w \MLmodels \metaB$.
	      % \answer{ NAME % collaborator names
	      % }{% body of the argument
	      %   Your answer goes here
	      % }

	\item $\M, w \MLmodels[] \Diamond\metaA$ \textit{iff} $\M, u \MLmodels \metaA$ for some $u \in W$ such that $R(w, u)$.
	      % \answer{ NAME % collaborator names
	      % }{% body of the argument
	      %   Your answer goes here
	      % }

	\item $\M, w \MLmodels[] \Box\metaA$ \textit{iff} $(w)_R \subseteq \interpret{\M}{\metaA}$.
	      % \answer{ NAME % collaborator names
	      % }{% body of the argument
	      %   Your answer goes here
	      % }

	\item $\M, w \MLmodels[] \Box(\metaA \rightarrow \metaB)$ \textit{iff} $\interpret{\M}{\metaA}^w \subseteq \interpret{\M}{\metaB}$.
	      % \answer{ NAME % collaborator names
	      % }{% body of the argument
	      %   Your answer goes here
	      % }

	\item $\M, w \MLmodels[] \Box(\metaA \leftrightarrow \metaB)$ \textit{iff} $\interpret{\M}{\metaA}^w = \interpret{\M}{\metaB}^w$.
	      % \answer{ NAME % collaborator names
	      % }{% body of the argument
	      %   Your answer goes here
	      % }

	\item $\M, w \MLmodels[] \Diamond\metaA$ \textit{iff} $(w)_R \cap \interpret{\M}{\metaA} \neq \varnothing$.
	      % \answer{ NAME % collaborator names
	      % }{% body of the argument
	      %   Your answer goes here
	      % }

\end{enumerate}





\section{Frames}

\begin{enumerate}

	\item[\bf Relations:] Evaluate the following, providing a proof or countermodel:

	\item Every serial frame is reflexive.
	      \answer{ Ben % Collaborators
		      % Reviewers
	      }{% body of the argument
		      \vspace{-.2in}
		      \begin{multicols}{2}
			      \noindent
			      Letting $W = \set{w_i : 0 \leq i \leq 4}$ where $R = \set{\tuple{w_x, w_y} : y = x + 1 ~~ (\bmod\ 5)}$, we may observe that every world sees some world but no world sees itself.
			      Thus $\tuple{W, R}$ is serial without being reflexive, providing a countermodel to the stated claim.

			      We may produce infinitely many countermodels by simply varying the cardinality of $\vert{W} > 1$.
			      \qedp  % The diamond is because countermodels are existential instead of universal like proofs.

			      \columnbreak

			      % NOTE: This is more elaborate than it needs to be, making extra work to write and to check.
			      % Instead it is better to use as simple of an example as you can find, adding complexity only when needed.
			      % I will put a simpler example below that you can adapt.
			      \begin{center}
				      \begin{tikzpicture}
					      % Define the five worlds in a circle
					      \node[world] (w0) at (72 :1.75) {$w_0$};
					      \node[world] (w1) at (360:1.75) {$w_1$};
					      \node[world] (w2) at (288:1.75) {$w_2$};
					      \node[world] (w3) at (216:1.75) {$w_3$};
					      \node[world] (w4) at (144:1.75) {$w_4$};

					      % Draw arrows clockwise between worlds
					      \draw[arrow] (w0) to[bend right] (w1);
					      \draw[arrow] (w1) to[bend right] (w2);
					      \draw[arrow] (w2) to[bend right] (w3);
					      \draw[arrow] (w3) to[bend right] (w4);
					      \draw[arrow] (w4) to[bend right] (w0);
				      \end{tikzpicture}
			      \end{center}
		      \end{multicols}

		      \noindent
		      NOTE: This solution is needlessly complex.
		      I use it as an example of what to avoid though it is a perfectly correct countermodel.
		      Instead, try to find the simplest countermodels that you can, adding complexity only as needed.

		      % NOTE: This example would not work as is, but could easily be adapted to show what the example above shows.
		      % \begin{center}
		      %   \begin{tikzpicture}
		      %     % Define the two worlds
		      %     \node[world] (w) at (0,0) {$w$};
		      %     \node[world] (u) at (2,0) {$u$};
		      %     
		      %     % Draw arrows
		      %     \draw[arrow] (w) to[loop left] (w);
		      %     \draw[arrow] (w) to (u);
		      %   \end{tikzpicture}
		      % \end{center}

	      }

	\item Every serial symmetric frame is reflexive.
	      % \answer{ NAME % collaborator names
	      % }{% body of the argument
	      %   Your answer goes here
	      % }

	\item How many frames are both symmetric and transitive but not reflexive.
	      % \answer{ NAME % collaborator names
	      % }{% body of the argument
	      %   Your answer goes here
	      % }

	\item Every frame that is left and right Euclidean is symmetric and transitive.
	      % \answer{ NAME % collaborator names
	      % }{% body of the argument
	      %   Your answer goes here
	      % }

	\item Every frame that is left and right Euclidean is symmetric and transitive.
	      % \answer{ NAME % collaborator names
	      % }{% body of the argument
	      %   Your answer goes here
	      % }

	\item Every frame that is symmetric and transitive is left and right Euclidean.
	      % \answer{ NAME % collaborator names
	      % }{% body of the argument
	      %   Your answer goes here
	      % }

	\item Every frame that is transitive and both left and right Euclidean is symmetric.
	      % \answer{ NAME % collaborator names
	      % }{% body of the argument
	      %   Your answer goes here
	      % }

	\item Every frame that is symmetric and left Euclidean is transitive.
	      \answer{ Ben % Collaborators
		      % Reviewers
	      }{% body of the argument
		      Let $\tuple{W, R}$ be a symmetric (\textsc{sym}) and left Euclidean (\textsc{leu}) frame.
		      Since $R$ is transitive (\textsc{tra}) if $R = \varnothing$, we may assume otherwise.
		      Letting $R(x, y)$ and $R(y, z)$ for arbitrary $x,y,z \in W$, we aim to show that $R(x, z)$.

		      Since $R(z, y)$ follows by \textsc{sym}, we know $R(x, z)$ by \textsc{leu}.
		      Generalizing on $x, y, z \in W$, we may conclude that $\tuple{W, R}$ is transitive as desired.
		      \qed
	      }

	\item There is a finite serial transitive frame that is neither reflexive nor symmetric.
	      % \answer{ NAME % collaborator names
	      % }{% body of the argument
	      %   Your answer goes here
	      % }

	\item A symmetric frame is left Euclidean just in case it is right Euclidean.
	      % \answer{ NAME % collaborator names
	      % }{% body of the argument
	      %   Your answer goes here
	      % }

	\item The relational image of a transitive, symmetric, reflexive frame is a partition.
	      % \answer{ NAME % collaborator names
	      % }{% body of the argument
	      %   Your answer goes here
	      % }

	\item Every total frame is a partition.
	      % \answer{ NAME % collaborator names
	      % }{% body of the argument
	      %   Your answer goes here
	      % }

	\item There is a symmetric total frame that is not a partition.
	      % \answer{ NAME % collaborator names
	      % }{% body of the argument
	      %   Your answer goes here
	      % }

	\item[\bf Countermodels:] Evaluate the following, providing a proof or countermodel.
	      If there is a countermodel, replace $K$ with the weakest set of constraints $C$ to make the claim hold.
	      You do not need to prove that it is the weakest set of constraints.

	\item $\Box \metaA \MLmodels[K] \Box\Box \metaA$
	      % \answer{ NAME % collaborator names
	      % }{% body of the argument
	      %   Your answer goes here
	      % }

	\item $\Box \metaA \MLmodels[K] \metaA$
	      % \answer{ NAME % collaborator names
	      % }{% body of the argument
	      %   Your answer goes here
	      % }

	\item $\metaA \MLmodels[K] \Box \metaA$
	      % \answer{ NAME % collaborator names
	      % }{% body of the argument
	      %   Your answer goes here
	      % }

	\item $\Box(\metaA \vee \metaB) \MLmodels[K] \Box \metaA \vee \Box \metaB$
	      % \answer{ NAME % collaborator names
	      % }{% body of the argument
	      %   Your answer goes here
	      % }

	\item $\Box \metaA \MLmodels[K] \Diamond \metaA$
	      % \answer{ NAME % collaborator names
	      % }{% body of the argument
	      %   Your answer goes here
	      % }

	\item $\Diamond \metaA \MLmodels[K] \Box \metaA$
	      % \answer{ NAME % collaborator names
	      % }{% body of the argument
	      %   Your answer goes here
	      % }

	\item $\Diamond\Box \metaA \MLmodels[K] \Box\Diamond \metaA$
	      % \answer{ NAME % collaborator names
	      % }{% body of the argument
	      %   Your answer goes here
	      % }

	\item $\Box\Diamond \metaA \MLmodels[K] \Diamond\Box \metaA$
	      % \answer{ NAME % collaborator names
	      % }{% body of the argument
	      %   Your answer goes here
	      % }

	\item $\Box \metaA \MLmodels[K] \Box\Diamond \metaA$
	      % \answer{ NAME % collaborator names
	      % }{% body of the argument
	      %   Your answer goes here
	      % }

	\item $\MLmodels[K] \neg\Box(\metaA \wedge \neg \metaA)$
	      % \answer{ NAME % collaborator names
	      % }{% body of the argument
	      %   Your answer goes here
	      % }

	\item $\Diamond \metaB \MLmodels[K] \neg\Box(\metaA \wedge \neg \metaA)$
	      % \answer{ NAME % collaborator names
	      % }{% body of the argument
	      %   Your answer goes here
	      % }

	\item $\Box \metaA \MLmodels[K] \Box(\metaB \rightarrow \metaA)$.
	      % \answer{ NAME % collaborator names
	      % }{% body of the argument
	      %   Your answer goes here
	      % }

	\item $\neg\Box \metaA \MLmodels[K] \Box(\metaA \rightarrow \metaB)$.
	      % \answer{ NAME % collaborator names
	      % }{% body of the argument
	      %   Your answer goes here
	      % }

\end{enumerate}






\section{Characterization}

\begin{enumerate}

	\item[\bf Semantic Proofs:] Provide semantic proofs of the following:

	\item If $\MLmodels[K] \metaA$, then $\MLmodels[K] \Box\metaA$
	      % \answer{ NAME % collaborator names
	      % }{% body of the argument
	      %   Your answer goes here
	      % }

	\item $\MLmodels[K] \Box(\metaA \rightarrow \metaB)\rightarrow(\Box\metaA \rightarrow \Box\metaB)$
	      % \answer{ NAME % collaborator names
	      % }{% body of the argument
	      %   Your answer goes here
	      % }

	\item $\MLmodels[D] \Box \metaA \rightarrow\Diamond \metaA$
	      % \answer{ NAME % collaborator names
	      % }{% body of the argument
	      %   Your answer goes here
	      % }

	\item $\MLmodels[T] \Box \metaA \rightarrow \metaA$
	      % \answer{ NAME % collaborator names
	      % }{% body of the argument
	      %   Your answer goes here
	      % }

	\item $\MLmodels[B] \metaA \rightarrow \Box\Diamond\metaA$
	      % \answer{ NAME % collaborator names
	      % }{% body of the argument
	      %   Your answer goes here
	      % }

	\item $\MLmodels[4] \Box\metaA \rightarrow \Box\Box\metaA$
	      % \answer{ NAME % collaborator names
	      % }{% body of the argument
	      %   Your answer goes here
	      % }

	\item $\MLmodels[5] \Diamond\metaA \rightarrow \Box\Diamond\metaA$
	      % \answer{ NAME % collaborator names
	      % }{% body of the argument
	      %   Your answer goes here
	      % }

	\item[\bf Equivalences:] Provide semantic proofs of the following equivalences:

	\item $\neg\Box \metaA \MLequiv[K] \Diamond\neg \metaA$.
	      % \answer{ NAME % collaborator names
	      % }{% body of the argument
	      %   Your answer goes here
	      % }

	\item $\neg\Diamond \metaA \MLequiv[K] \Box\neg \metaA$.
	      % \answer{ NAME % collaborator names
	      % }{% body of the argument
	      %   Your answer goes here
	      % }

	\item $\Diamond(\metaA \vee \metaB) \MLequiv[K]  \Diamond\metaA \vee \Diamond\metaB$
	      % \answer{ NAME % collaborator names
	      % }{% body of the argument
	      %   Your answer goes here
	      % }

	\item $\Diamond\Box \metaA \MLequiv[B] \Diamond\Box\Diamond\Box\metaA$
	      % \answer{ NAME % collaborator names
	      % }{% body of the argument
	      %   Your answer goes here
	      % }

	\item $\Box\Box \metaA \MLequiv[4] \Box \metaA$.
	      % \answer{ NAME % collaborator names
	      % }{% body of the argument
	      %   Your answer goes here
	      % }

	\item $\Box\Diamond \metaA \MLequiv[5] \Diamond \metaA$.
	      % \answer{ NAME % collaborator names
	      % }{% body of the argument
	      %   Your answer goes here
	      % }

	\item $\Diamond\Box \metaA \MLequiv[5] \Box \metaA$.
	      % \answer{ NAME % collaborator names
	      % }{% body of the argument
	      %   Your answer goes here
	      % }

\end{enumerate}


%%% Bibliography %%%

% \vfill
% \begin{small} %%Makes bib small text size
%   \singlespacing %%Makes single spaced
%   \bibliographystyle{../../assets/bib_style} %%bib style found locally or in textmf/bibtex/bst
%   \setlength{\bibsep}{0.5pt} %%Changes spacing between bib entries
%   \bibliography{../../assets/modal_history} %%bib database found locally or in textmf/bibtex/bib
%   \thispagestyle{empty} %%Removes page numbers
% \end{small} %%End makes bib small text size

\end{document}
