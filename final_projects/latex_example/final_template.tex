%%%%%%%%%%%%%%%%%%%%%%%%%%%%%%%%%%%%%%%%%%%%%%%%%%%%%%%%%%%%%%%%%
%% Final Project Template
%% Course: The Modern History of Modal Logic at MIT
%% Last Updated: May 2025
%%%%%%%%%%%%%%%%%%%%%%%%%%%%%%%%%%%%%%%%%%%%%%%%%%%%%%%%%%%%%%%%%

\documentclass[11pt, a4paper]{article}  % Document class with paper and font size

%% Import style files - problem_set.sty contains most formatting
\usepackage{assets/formatting}      % Main styling package
\usepackage{assets/notation}         % Mathematical notation

%% Document information
\title{Project Title}                                % Document title
\class{The Modern History of Modal Logic}             % Document title
\author{Your Name}                                    % Author name (optional)
\pset{--- Final Project ---}                                  % Course/assignment information
\date{\today}                                         % Current date
\setheader{MIT, The Modern History of Modal Logic}    % Set page header



%%%%%%%%%%%%%%%%%%%%%%%%%%%%%%%%%%%%%%%%%%%%%%%%%%%%%%%%%%%%%%%%%
%% INSTRUCTIONS
%%%%%%%%%%%%%%%%%%%%%%%%%%%%%%%%%%%%%%%%%%%%%%%%%%%%%%%%%%%%%%%%%

% Please maintain the convention of one sentence for each numbered line.
% Cite any texts or textbooks you referred to throughout.
% Good to leave white space between sections to improve readability



%%%%%%%%%%%%%%%%%%%%%%%%%%%%%%%%%%%%%%%%%%%%%%%%%%%%%%%%%%%%%%%%%
\begin{document}
%%%%%%%%%%%%%%%%%%%%%%%%%%%%%%%%%%%%%%%%%%%%%%%%%%%%%%%%%%%%%%%%%

%% Print title
\papertitle

\begin{abstract}
\noindent
  Your abstract goes here.
  This should be a brief summary of your project, outlining the problem and addressed and key findings.
  (3 - 5 sentences)
\end{abstract}
\vspace{20pt}



%%%%%%%%%%%%%%%%%%%%%%%%%%%%%%%%%%%%%%%%%%%%%%%%%%%%%%%%%%%%%%%%%
%% MAIN CONTENT
%%%%%%%%%%%%%%%%%%%%%%%%%%%%%%%%%%%%%%%%%%%%%%%%%%%%%%%%%%%%%%%%%

\hypsection{Introduction}
  \label{sec:intro}

% Your content goes here

You may want to cite \citet{Carnap1947}, or include the page number where \citet[p.~7]{Prior1967} credits \citepos{Mctaggart1908} (passive) argument that time is unreal, but only if relevant.





\hypsection{Using Formal Notation}
  \label{sec:notation}

%% MATHEMATICAL NOTATION
% The notation.sty package provides many useful symbols
% For example, you can use \always for the box operator and \sometimes for the diamond

This is similar to the problem sets.
For instance, $\Box \varphi \rightarrow \varphi$ is the T axiom.
% You can also reference specific modal systems
Use the appropriate commands defined in \texttt{notation.sty} such as $\K$, or define new commands as convenient.
% Additional symbols for other logics are also available
In tense logic, we use $\past$ for ``it was the case that'' and $\future$ for ``it is going to be the case that''.
And so on.





\hypsection{Using Theorem Environments}
  \label{sec:theorems}

%% DEFINITION ENVIRONMENT
% Use the Dthm environment for formal definitions
% The definition number is automatically generated as Definition section.number

\begin{Dthm}{(Kripke Frame)} \label{def:kripke}
  A \emph{Kripke frame} is a pair $\F = \tuple{\W, R}$ where:
  \begin{enumerate}
    \item $\W$ is a non-empty set of possible worlds
    \item $R \subseteq \W \times \W$ is an accessibility relation on $\W$
  \end{enumerate}
\end{Dthm}

% You can reference definitions elsewhere in your text:
\noindent
As we can see in \ref{def:kripke}, a Kripke frame consists of two components.




%% LEMMA ENVIRONMENT  
% Use the Lthm environment for lemmas that support your theorems
% The lemma number is automatically generated as Lemma section.number

\begin{Lthm} \label{lem:minor} % Always add a label for cross-referencing
  A well-formed formula $\metaA$ is a theorem of $\K$ \textit{iff} \ldots
\end{Lthm}

\begin{proof}
  This is a proof of a lemma.

\end{proof}

% You can cite lemmas using \ref:
\ref{lem:minor} establishes an important cornerstone of...




%% THEOREM ENVIRONMENT
% Use the Tthm environment for theorems
% The theorem number is automatically generated as Theorem section.number

\begin{Tthm}{(Optional Name)} \label{thm:example} % Always add a label for cross-referencing
  For any modal system $\K$, if $\metaA$ is a theorem, then $\Box\metaA$ is also a theorem.
\end{Tthm}

\begin{proof}
  We know by \ref{lem:minor}...

\end{proof}

% To reference the theorem elsewhere in your text:
As shown in \ref{thm:example}, the necessitation rule is fundamental to normal modal systems.




%% COROLLARY ENVIRONMENT
% Use the Cthm environment for results that follow directly from a theorem
% The corollary number is automatically generated as Corollary section.number

\begin{Cthm} \label{cor:s5}
  In the modal system $\Sfive$, if $\Diamond\metaA$ is true at some world, then \ldots
\end{Cthm}




%% RULE ENVIRONMENT
% Use the Rthm environment for inference rules
% The rule number is automatically generated as Rule number

\begin{Rthm}{(Some Inference)} \label{rule:mp}
  From $\metaA$ and $\metaA \eif \metaB$, we may infer $\metaB$.
\end{Rthm}

% You can create a proof using the hilbert environment
\begin{hilbert}
  \from{Premise}{p \eand q}
  \from{Premise}{p \eif r}
  \from{$\eand$-Elimination}{p}
  \from{Modus Ponens}{r}
\end{hilbert}




\hypsection{Modal Logic Diagrams}
  \label{sec:diagrams}

% This section demonstrates how to create Kripke frame diagrams using TikZ

% The notation.sty file includes useful TikZ configurations for modal logic diagrams
% World styling and arrow styling are predefined

%% Basic reflexive world diagram (System T)
\begin{figure}[h]
  \centering
  \begin{tikzpicture}
    % Create a world with reflexive arrow
    \node[world] (w) at (0,0) {$w$};
    \draw[arrow, loop above] (w) to (w);
    
    % Add a label for the model
    \node[above=0.5cm of w] {System T: Reflexive Accessibility};
  \end{tikzpicture}
  \caption{A model with a single reflexive world, characteristic of System T.}
  \label{fig:t-model}
\end{figure}

%% S4 model diagram (transitive relation)
\begin{figure}[h]
  \centering
  \begin{tikzpicture}
    % Create three worlds in a row
    \node[world] (w1) at (0,0) {$w_1$};
    \node[world] (w2) at (2,0) {$w_2$};
    \node[world] (w3) at (4,0) {$w_3$};
    
    % Add arrows showing accessibility
    \draw[arrow] (w1) to (w2);
    \draw[arrow] (w2) to (w3);
    \draw[arrow, bend right=40] (w1) to (w3); % Transitive arrow
    
    % Add reflexive arrows
    \draw[arrow, loop above] (w1) to (w1);
    \draw[arrow, loop above] (w2) to (w2);
    \draw[arrow, loop above] (w3) to (w3);
    
    % Add a label for the model
    \node[below=0.5cm of w2] {System S4: Reflexive and Transitive Accessibility};
  \end{tikzpicture}
  \caption{A model with transitive accessibility relation, characteristic of System S4.}
  \label{fig:s4-model}
\end{figure}

%% S5 model diagram (equivalence relation)
\begin{figure}[h]
  \centering
  \begin{tikzpicture}
    % Create three worlds in a triangle
    \node[world] (w1) at (0,0) {$w_1$};
    \node[world] (w2) at (3,0) {$w_2$};
    \node[world] (w3) at (1.5,2) {$w_3$};
    
    % Add two-way arrows between all worlds (symmetry)
    \draw[arrow, bend left=15] (w1) to (w2);
    \draw[arrow, bend left=15] (w2) to (w1);
    \draw[arrow, bend left=15] (w1) to (w3);
    \draw[arrow, bend left=15] (w3) to (w1);
    \draw[arrow, bend left=15] (w2) to (w3);
    \draw[arrow, bend left=15] (w3) to (w2);
    
    % Add reflexive arrows (optional - often omitted in S5 diagrams)
    % \draw[arrow, loop left] (w1) to (w1);
    % \draw[arrow, loop right] (w2) to (w2);
    % \draw[arrow, loop above] (w3) to (w3);
    
    % Add a label for the model
    \node[above=0.5cm of w3] {System S5: Equivalence Relation};
  \end{tikzpicture}
  \caption{A model where all worlds access each other, characteristic of System S5.}
  \label{fig:s5-model}
\end{figure}

%% Branching time model
\begin{figure}[h]
  \centering
  \begin{tikzpicture}
    % Create worlds representing timestamps
    \node[world] (past) at (0,0) {$t_0$};
    \node[world] (present) at (2,0) {$t_1$};
    \node[world] (future1) at (4,1) {$t_2a$};
    \node[world] (future2) at (4,-1) {$t_2b$};
    
    % Add arrows showing temporal accessibility
    \draw[arrow] (past) to (present);
    \draw[arrow] (present) to (future1);
    \draw[arrow] (present) to (future2);
    
    % Add a label for the model
    \node[below=1.25cm of present] {Branching Time Model};
  \end{tikzpicture}
  \caption{A branching time model with a single past and multiple possible futures.}
  \label{fig:branching-time}
\end{figure}

% Example of referencing a figure
As shown in Figure \ref{fig:s5-model}, System S5 can be represented as a fully connected graph where each world is accessible from every other world.




%%%%%%%%%%%%%%%%%%%%%%%%%%%%%%%%%%%%%%%%%%%%%%%%%%%%%%%%%%%%%%%%%
%% BIBLIOGRAPHY
%%%%%%%%%%%%%%%%%%%%%%%%%%%%%%%%%%%%%%%%%%%%%%%%%%%%%%%%%%%%%%%%%

\newpage
\begin{small}                                          % Smaller text for bibliography
  \singlespacing                                       % Single-spaced bibliography
  \bibliographystyle{assets/bib_style}              % Bibliography style
  \setlength{\bibsep}{0.5pt}                           % Spacing between entries
  \bibliography{assets/modal_history}               % Bibliography database
  \thispagestyle{empty}                                % No page number
\end{small}

\end{document}
